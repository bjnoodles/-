\documentclass{BHCexam}
%\usepackage[links]{hyperref}
\usepackage{setspace}
\begin{document}
	\biaoti{新课标习题分类}
	\fubiaoti{}
	\maketitle
	\tableofcontents
	\section{函数}
\begin{questions}
	\qs 函数$f(x)$在$ \left(-\infty,+\infty\right) $单调递减,且为奇函数,若$ f(1)=-1 $,则满足$ -1\le f(x-2)\le1 $的$ x $的取值范围是\xx
	\onech{$ \left[-2,2\right]$}{$ \left[-1,1\right]$}{$ \left[0,4\right]$}{$ \left[1,3\right]$}

	\qs  设$ x,y,z $为正数,且$ 2^x=3^y=5^z $,则\xx
	\onech{$ 2x<3y<5z$}{$ 5z<2x<3y$}{$ 3y<5z<2x$}{$ 3y<2x<5z$}
	\qs 已知函数$f(x)=\ln x +\ln(2-x)$,则\xx
	\twoch{$f(x)$在$ \left(0,2\right) $上单调递增}{$ f(x)$在$ \left(0,2\right) $上单调递减}{$ y=f(x)$的图像关于直线$ x=1 $对称}{$y=f(x) $的图像关于点$ \left(1,0\right) $对称}
	\qs 函数$f(x)=\ln\left(x^2-2x-8\right)$的单调递增区间是\xx
	\onech{$ \left(-\infty,-2\right)$}{$ \left(-\infty,-1\right)$}{$ \left(1,+\infty\right)$}{$ \left(4,+\infty\right)$}
		\qs 设函数$f(x)=\Bigg\{\begin{aligned}
	&x+1,&x\le0,\\
	&2^x,&x>0.
	\end{aligned}$则满足$ f(x)+f(x-\dfrac{1}{2})>1 $的$x$的取值范围是\tk.
	\qs 已知函数$f(x)$是定义在$ \mathbf{R} $上的奇函数,当$ x\in\left(-\infty,0\right) $时,,$ f(x)=2x^3+x^2 $,则$ f(2)=\tk. $
	\qs 若$ a>b>1,~0<c<1 $,则\xx
	\twoch{$ a^c<b^c$}{$ ab^c<ba^c$}{$ a\log_bc<b\log_ac$}{$ \log_ac<\log_bc$}
	\qs 已知$ a=2^{\frac{4}{3}} ,b=4^{\frac{2}{5}},c=25^{\frac{1}{3}}$,则\xx
	\onech{$ b<a<c$}{$ a<b<c$}{$ b<c<a$}{$ c<a<b$}
  	\qs 从区间$ \left[0,1\right] $随机抽取$ 2n $个数$ x_1,x_2,\cdots,x_n,y_1,y_2,\cdots,y_n $,构成$ n $个数对$\left(x_1,y_1\right)  ,\left(x_2,y_2\right),\cdots,\left(x_n,y_n\right)$,其中两数的平方和小于$ 1 $的数对共有$ m $个,则用随机模拟的方法得到的圆周率$ \pi $的近似值为\xx
  	\onech{$ \dfrac{4n}{m}$}{$ \dfrac{2n}{m}$}{$ \dfrac{4m}{n}$}{$ \dfrac{2m}{n}$}
  	\question
  	若函数$f(x)=x-\dfrac{1}{3}\sin 2x+a\sin x$在$(-\infty,+\infty)$单调递增,则$a$的取值范围是\xx
  	\onech{$\left[-1,1\right]$}{$\left[-1,\dfrac{1}{3}\right]$}{$\left[-\dfrac{1}{3},\dfrac{1}{3}\right]$}{$\left[-1,-\dfrac{1}{3}\right]$}
  	\qs 已知函数$f(x)~(x\in \mathbf{R})$满足$f(-x)=2-f(x)$,若函数$y=\dfrac{x+1}{x}$与$y=f(x)$图象的交点为$(x_1,y_1),(x_2,y_2),$\\$\cdots,(x_m,y_m)$,则$\sum\limits_{i=1}^{m}(x_i+y_i)=$\xx
  	\onech{$0$}{$m$}{$2m$}{$4m$}
  	\qs 已知函数$f(x)\left(x\inR\right)$满足$ f(x)=f(2-x) $,若函数$ y=\abs{x^2-2x-3} $与$ y=f(x) $图像的交点为$ \left(x_1,y_1\right),\left(x_2,y_2\right) ,\cdots\left(x_n,y_n\right)$,则$ \sum\limits_{i=1}^m x_i=$\xx
  	\onech{$ 0$}{$ m$}{$ 2m$}{$ 4m$}
  	\qs 设函数$f(x)=\left\{\begin{aligned}
  	&1+\log_2\left({2-x}\right),&x<1\\
  	&2^{x-1},&x\ge1.
  	\end{aligned}\right.$,$ f(-2)+f\left(\log_2{12}\right)= $\xx
  	\onech{$ 3$}{$ 6$}{$ 9$}{$ 12$}
  	\qs 已知$f(x)$为偶函数,当$ x<0 $时,$f(x)=\ln (-x)+3x,~$则曲线$ y=f(x) $在点$ \left(1,-3\right) $处的切线方程是\tk.
  	\qs 若$f(x)=x\ln (x+\sqrt{a+x^2})$为偶函数,则$ a= $\tk.
\end{questions}




\section{三角函数}
\begin{questions}
	\qs 已知曲线$ C_1:y=\cos x ,~C_2:y=\sin\left(2x+\dfrac{2\pi}{3}\right)$,则下面结论正确的是\xx
	\fourch{把$C_1$上各点的横坐标伸长到原来的$2$倍,纵坐标不变,再把得到的曲线向右平移$ \dfrac{\pi}{6} $个单位长度,得到曲线$C2$}{把$C_1$上各点的横坐标伸长到原来的$2$倍,纵坐标不变,再把得到的曲线向右平移$ \dfrac{\pi}{12} $个单位长度,得到曲线$C_2$}{把$C_1$上各点的横坐标缩短到原来的$\dfrac{1}{2}$倍,纵坐标不变,再把得到的曲线向右平移$ \dfrac{\pi}{6} $个单位长度,得到曲线$C_2$}{把$C_1$上各点的横坐标缩短到原来的$\dfrac{1}{2}$倍,纵坐标不变,再把得到的曲线向右平移$ \dfrac{\pi}{12} $个单位长度,得到曲线$C_2$}
	\qs 设函数$f(x)=\cos\left(x+\dfrac{\pi}{3}\right)$,则下列结论错误的是\xx
	\twoch{$ f(x)$的一个周期为$ -2\pi $}{$ y=f(x)$的图像关于直线$ x=\dfrac{8\pi}{3} $对称}{$f(x+\pi) $的一个零点为$x=\dfrac{\pi}{6}$}{$ f(x)$在$ \left(\dfrac{\pi}{2},\pi\right) $单调递减}
	\qs $\triangle ABC$的内角$ A,~B,~C $的对边分别为$ a,~b,~c $,已知$ \sin B+\sin A\left(\sin C-\cos C\right)=0,~a=2,c=\sqrt{2},~ $则$ C= $\xx.
	\onech{$ \dfrac{\pi}{12}$}{$ \dfrac{\pi}{6}$}{$ \dfrac{\pi}{4}$}{$ \dfrac{\pi}{3}$}
	\qs 已知$ \sin \alpha -\cos \alpha =\dfrac{4}{3} ,~$则$ \sin 2\alpha =$\xx
	\onech{$ -\dfrac{7}{9}$}{$ -\dfrac{2}{9}$}{$ \dfrac{2}{9}$}{$\dfrac{7}{9} $} 
	\qs 若$ \tan \alpha=\dfrac{3}{4},~ $则$ \cos^2\alpha+2\sin2\alpha= $\xx
	\onech{$ \dfrac{64}{25}$}{$ \dfrac{48}{25}$}{$ 1$}{$ \dfrac{16}{25}$}
	\qs 函数$f(x)=\dfrac{1}{5}\sin\left(x+\dfrac{\pi}{3}\right)+\cos\left(x-\dfrac{\pi}{6}\right)$的最大值为\xx
	\onech{$ \dfrac{6}{5}$}{$ 1$}{$ \dfrac{3}{5}$}{$\dfrac{1}{5} $}
	\question 已知函数$f(x)=\sin(\omega x+\varphi)~\left(\omega>0,\abs{\varphi}\le \dfrac{\pi}{2}\right),x=-\dfrac{\pi}{4}$为$f(x)$的零点,$x=\dfrac{\pi}{4}$为$y=f(x)$图象的对称轴,且$f(x)$在$\left(\dfrac{\pi}{18},\dfrac{5\pi}{36}\right)$单调,则$\omega$的最大值为\xx
	\onech{11}{9}{7}{5}
	\qs 若$ \cos \left(\dfrac{\pi}{4}-\alpha\right)=\dfrac{3}{5} $,则$ \sin2\alpha= $\xx
	\onech{$ \dfrac{7}{25}$}{$ \dfrac{1}{5}$}{$ -\dfrac{1}{5}$}{$ -\dfrac{7}{25}$}
	\qs 若$ \tan\theta=\dfrac{1}{3} $,则$ \cos2\theta= $\xx
	\onech{$ -\dfrac{4}{5}$}{$ -\dfrac{1}{5}$}{$\dfrac{1}{5} $}{$\dfrac{4}{5} $}
	\qs 函数$f(x)=\cos 2x+6\cos\left(\dfrac{\pi}{2}-x\right)$的最大值为\xx
	\onech{$ 4$}{$ 5$}{$ 6$}{$ 7$}
	\qs 在$\triangle ABC$中,$ B=\dfrac{\pi}{4} ,~BC$边上的高等于$ \dfrac{1}{3} BC$,则$ \sin A= $\xx
	\onech{$ \dfrac{3}{10}$}{$ \dfrac{\sqrt{10}}{10}$}{$ \dfrac{\sqrt{5}}{5}$}{$ \dfrac{3\sqrt{10}}{10}$}
	\qs 已知$ \theta $是第四象限角,且$\sin\left(\theta+\dfrac{\pi}{4}\right)=\dfrac{3}{5} $,则$ \tan\left(\theta-\dfrac{\pi}{4}\right)= $\tk.
	\qs 函数$f(x)=2\cos x+\sin x$的最大值为\tk.
	\qs $\triangle ABC$的内角$ A,\ B,\ C $的对边分别为$ a,~b,~c $,若$ 2b\cos B=a\cos C+c\cos A $,则$ B=\tk. $
	\qs 函数$f(x)=\sin ^2x +\sqrt{3}\cos x-\dfrac{3}{4}, \left(x\in \left[0,\dfrac{\pi}{2}\right]\right)$的最大值是\tk.
	\qs 已知$ \alpha\in\left(0,\dfrac{\pi}{2}\right),~\tan\alpha=2 $,则$ \cos\left(\alpha-\dfrac{\pi}{4} \right)$=\tk.
	\qs $\triangle ABC$的内角$ A,~B,~C $的对边分别为$ a,~b,~c $,若$ \cos A=\dfrac{4}{5} ,~\cos C=\dfrac{5}{13},~a=1.$则$ b=\tk. $
	\question 在平面四边形$ABCD$中,$\angle A=\angle B=\angle C=75^{\circ},~$$ BC=2,~ $则$AB$的取值范围是\tk.
	\qs 函数$ y=\sin x-\sqrt{3}\cos x $的图像可由函数$ y=\sin x+\sqrt{3}\cos x $的图像至少向右平移\tk 个单位长度得到.
	\qs $\triangle ABC$的内角$ A,~B,~C $的对边分别为$ a,~b,~c $,已知$\triangle ABC$的面积为$\dfrac{a^2}{3\sin A}$.
 	\begin{parts}
 		\part 求$\sin B\sin C.$
 		\part 若$ 6\cos B\cos C=1,a=3. $求$\triangle ABC$的周长.
 	\end{parts}
 \qs $\triangle ABC$的内角$ A,~B,~C $的对边分别为$ a,~b,~c $,已知$ \sin\left(A+C\right)=8\sin^2\dfrac{B}{2} .$
 \begin{parts}
 	\part 求$ \cos B $;
 	\part 若$ a+c=6,\triangle ABC $的面积为$ 2 $,求$ b .$
 \end{parts}
\qs $\triangle ABC$的内角$ A,~B,~C $的对边分别为$ a,~b,~c $,已知$ \sin A+\sqrt{3}\cos A=0,~a=2\sqrt{7},b=2 $.
\begin{parts}
	\part 求$ c $;
	\part 设$ D $为$ BC $边上的一点,且$ AD\bm{\bot}AC $,求$\triangle ABD$的面积.
\end{parts}
\qs $\triangle ABC$的内角$ A,B,C $的对边$ a,b,c ~$.已知$ 2\cos C(a\cos B +b \cos A)=c $.
\begin{parts}
	\part 求$ C $;
	\part 若$ c=\sqrt{7} $,$ \triangle ABC $的面积为$ \dfrac{3\sqrt{3}}{2} $,求$ \triangle ABC $的周长.
\end{parts}
\qs 在$\triangle ABC$中,$ D $是$ BC $上的点,$ AD $平分$ \angle BAC ,~\triangle ABD$是$ \triangle ADC $面积的两倍.
\begin{parts}
	\part 求$ \dfrac{\sin \angle B}{\sin \angle C} $;
	\part 若$ AD=1,~DC=\dfrac{\sqrt{2}}{2} ,~$求$BD$和$ AC $的长.
\end{parts}
\end{questions}

\section{向量}
\begin{questions}
	\qs 已知$\triangle ABC$是边长为$2$的等边三角形,$P$为平面$ABC$内一点,则$\vv{PA}\cdot\left(\vv{PB}+\vv{PC}\right)$的最小值是\xx
	\onech{$ -2$}{$ -\dfrac{3}{2}$}{$ -\dfrac{4}{3}$}{$ -1$}
	\qs 在矩形$ ABCD $中,$ AB=1,~AD=2 $,动点$ P $在以点$C $为圆心且与$ BD $相切的圆上,若$ \vv{AP}=\lambda\vv{AB}+\mu\vv{AD} ,$则$ \lambda+\mu $的最大值为\xx
	\onech{$ 3$}{$ 2\sqrt{2}$}{$ \sqrt{5}$}{$ 2$}
	\qs 设向量$\bm{a}$,$\bm{b}$满足$ \abs{\bm{a+b}}=\sqrt{10},~\abs{\bm{a-b}}=\sqrt{6} $,则$ \bm{a\cdot b}= $\xx
	\onech{$ 1$}{$ 2$}{$ 3$}{$ 5$}
	\qs 已知向量$ \vv{BA}=\left(\dfrac{1}{2},\dfrac{\sqrt{3}}{2}\right),~\vv{BC}=\left(\dfrac{\sqrt{3}}{2},\dfrac{1}{2}\right) ,~$则$ \angle ABC= $\xx
	\onech{$ 30^{\circ}$}{$ 45^{\circ}$}{$ 60^{\circ}$}{$ 120^{\circ}$}
	\qs 设非零向量$ \bm{a,~b} $满足$ \abs{\bm{a+b}}=\abs{\bm{a-b}} $,则\xx
	\onech{$ \bm{a}\bot\bm{b}$}{$ \abs{\bm{a}}=\abs{\bm{b}}$}{$ \bm{a}\sslash\bm{b}$}{$ \bm{a}>\bm{b}$}
	\qs 设$ D $是$\triangle ABC$所在平面内一点,$ \vv{BC}=3\vv{CD} $,则\xx
	\twoch{$ \vv{AD}=-\dfrac{1}{3}\vv{AB}+\dfrac{4}{3}\vv{AC}$}{$ \vv{AD}=\dfrac{1}{3}\vv{AB}-\dfrac{4}{3}\vv{AC}$}{$\vv{AD}=\dfrac{4}{3}\vv{AB}+\dfrac{1}{3}\vv{AC} $}{$ \vv{AD}=\dfrac{4}{3}\vv{AB}-\dfrac{1}{3}\vv{AC}$}
	\qs 已知向量$\bm{ a},\bm{b}$的夹角为$ 60^{\circ} ,~\abs{\bm{a}}=2,\abs{\bm{b}}=1$,则$ \abs{\bm{a+2b}}= $\tk.
	\qs 设向量$ \vv{a},~\vv{b} $不平行,向量$ \lambda\vv{a}+\vv{b} $与$ \vv{a}+2\vv{b} $平行, 则实数$ \lambda=\tk. $
\end{questions}




\section{圆锥曲线}
\begin{questions}
	\qs 若双曲线$C:\dfrac{x^2}{a^2}-\dfrac{y^2}{b^2}=1~(a>0,b>0)$的一条渐近线被圆$ \left(x-2\right)^2+y^2=4 $所截得的弦长为$2$,则$ C $的离心率为\xx
	\onech{$ 2$}{$ \sqrt{3}$}{$ \sqrt{2}$}{$ \dfrac{2\sqrt{3}}{3}$}
	\qs 已知$F$为抛物线$C:y^2=4x$的焦点,过F作两条互相垂直的直线$l_1,l_2$,直线$l_1$与$C$交于$A,B$两点,直线$l_2$与$C$交于$D,E$两点,则$\abs{AB}+\abs{DE}$的最小值为\xx
	\onech{$ 16$}{$ 14$}{$ 12$}{$ 10$}
	\qs 已知椭圆$C$:$\dfrac{x^2}{a^2}+\dfrac{y^2}{b^2}=1~(a>b>0)$的左、右顶点分别为$A_1,A_2$,且以线段$ A_1,~A_2 $为直径的圆于直线$ bx-ay+2ab=0 $相切,则$ C $的离心率为\xx
	\onech{$ \dfrac{\sqrt{6}}{3}$}{$ \dfrac{\sqrt{3}}{3}$}{$ \dfrac{\sqrt{2}}{3}$}{$ \dfrac{1}{3}$}
	\question
	已知$O$为坐标原点,$F$是椭圆$C$:$\dfrac{x^2}{a^2}+\dfrac{y^2}{b^2}=1~(a>b>0)$的左焦点,$A$,$B$分别是$C$的左、右顶点.~$P$为$C$上一点,且$PF\bot x$轴,且过点$A$的直线$l$与线段$PF$交于点$M$,与$y$轴交于点$E$,若直线$BM$经过$OE$的中点,则$C$的离心率为\xx 
	\onech{$\dfrac13$}{$\dfrac12$}{$\dfrac23$}{$\dfrac34$}
	\qs 设$ A,~B $是椭圆$ C:\dfrac{x^2}{3}+\dfrac{y^2}{m}=1 $长轴的两个端点,若$ C $上存在点$ M $满足$ \angle AMB=120^{\circ},~ $则$ m $的取值范围是\xx
	\twoch{$ \left(0,1\right]\cup\left[9,+\infty\right)$}{$ \left(0,\sqrt{3}\right]\cup\left[9,+\infty\right)$}{$\left(0,1\right]\cup\left[4,+\infty\right) $}{$\left(0,\sqrt{3}\right]\cup\left[4,+\infty\right) $}
	\qs 设点$ M(x_0,1) $,若在圆$ O:x^2+y^2=1 $上存在点$ N $,使得$ \angle OMN=45^{\circ} $,则$ x_0 $的取值范围是\xx
	\onech{$ \left[-1,1\right]$}{$ \left[-\dfrac{1}{2},\dfrac{1}{2}\right]$}{$ \left[-\sqrt{2},\sqrt{2}\right]$}{$ \left[-\dfrac{\sqrt{2}}{2},\dfrac{\sqrt{2}}{2}\right]$}
	

	\qs 过抛物线$ C:y^2=4x $的焦点$F$,且斜率为$ \sqrt{3} $的直线交$C$于点$M \left(M\text{在}x\text{轴上方}\right)$,$l$为$C$的准线,点$N$在$l$上,且$MN\bot l$,则$M$到直线$NF$的距离为\xx
	\onech{$ \sqrt{5}$}{$ 2\sqrt{2}$}{$ 2\sqrt{3}$}{$ 3\sqrt{3}$}
	\qs 已知$ M\left(x_0,y_0\right) $是双曲线$ C:\dfrac{x^2}{2} -y^2=1$上的一点,$ F_1,~F_2 $是$ C $上的两个焦点,若$ \vv{MF_1}\cdot\vv{MF_2}<0 $,则$ y_0 $的取值范围是\xx
\begin{spacing}{1.4 }
		\twoch{$ \left(-\dfrac{\sqrt{3}}{3},\dfrac{\sqrt{3}}{3}\right)$}{$ \left(-\dfrac{\sqrt{3}}{6},\dfrac{\sqrt{3}}{6}\right)$}{$\left(-\dfrac{2\sqrt{2}}{3},\dfrac{2\sqrt{2}}{3} \right)$}{$ \left(-\dfrac{2\sqrt{3}}{3},\dfrac{2\sqrt{3}}{3} \right)$}
\end{spacing}

	\qs 已知$ A,~B $为双曲线$ E $的左、右顶点,点$ M $在$ E $上,$ \triangle ABM $为等腰三角形,且顶角为$ 120^{\circ} $,则$ E $的离心率为\xx
	\onech{$ \sqrt{5}$}{$ 2$}{$ \sqrt{3}$}{$ \sqrt{2}$}
	\qs 已知方程$ \dfrac{x^2}{m^2+n} -\dfrac{y^2}{3m^2-n}=1$表示双曲线,且该双曲线两焦点间的距离为$ 4 $,则$ n $的取值范围是\xx
	\onech{$ \left(-1,3\right)$}{$ \left(-1,\sqrt{3}\right)$}{$ \left(0,3\right)$}{$ \left(0,\sqrt{3}\right)$}
\qs 设$ F $为抛物线$ C:y^2=4x $的焦点,曲线$ y=\dfrac{k}{x}\left(k>0\right) $与$ C $交于点$ P ,~PF\bot x$轴,则$ k= $\tk.
	\qs 已知双曲线$C:\dfrac{x^2}{a^2}-\dfrac{y^2}{b^2}=1~(a>0,b>0)$的右顶点为$A$,以$A$为圆心,$b$为半径做圆$A$,圆$A$与双曲线$C$的一条渐近线交于$M,N$两点.若$\angle MAN=60^{\circ}$,则$C$的离心率为\tk.
	\qs 已知直线$ l:mx+y+3m-\sqrt{3}=0 $与圆$ x^2+y^2=12 $交于$ A,~B $两点,过$ A,~B $分别做$ l $的垂线与$ x $轴交于$ C,~D $两点,若$ AB=2\sqrt{3} $,则$ \abs{CD} =$\tk.
	\qs 已知$ F $是抛物线$ C:y^2=8x $的焦点,$ M $是$ C $上一点,$ FM $的延长线交$ y $轴于点$ N $.若,$ M $为$ FN $的中点,则$\abs{FN}=\tk$.
	\qs 已知椭圆$C$:$\dfrac{x^2}{a^2}+\dfrac{y^2}{b^2}=1~(a>b>0)$,四点$ P_1(1,1),P_2(0,1),P_3\left(-1,\dfrac{\sqrt{3}}{2}\right) ,P_4\left(1,\dfrac{\sqrt{3}}{2}\right)$中恰有三点在椭圆$ C $上.
	\begin{parts}
		\part 求$ C $的方程;
		\part 设直线$ l $不经过$ P_2 $点且与$ C $相交于点$ A,B $两点,若直线$ P_2A $与直线$ P_2B $的斜率的和为$ -1 $,证明:$ l $过定点.
	\end{parts}
\qs 设$O$为坐标原点,动点$M$在椭圆$C:\dfrac{x^2}{2}+y^2=1$上,过$M$做$x$轴的垂线,垂足为$N$,点$P$满足$ \vv{NP}=\sqrt{2}\vv{NM} $.
\begin{parts}
	\part 求点$ P $的轨迹方程;
	\part 设点$Q $在直线$ x=-3 $上,且$ \vv{OP}\cdot\vv{PQ}=1. $证明:过点$ P $且垂直于$ OQ $的直线$ l $过$ C $的左焦点$ F $.
\end{parts}
\qs 已知抛物线$C:y^2=2x$,过点$(2,0)$的直线$l$交$C$与$A,B$两点,圆$M$是以线段$AB$为直径的圆.
\begin{parts}
	\part 证明:坐标原点$O$在圆$M$上;
	\part 设圆$M$过点$P\left(4,-2\right)$,求直线$l$与圆$M$的方程.
\end{parts}
\qs 设$ A,B $为曲线$ C:y=\dfrac{x^2}{4} $上两点,$ A $与$ B $的横坐标之和为$ 4. $
\begin{parts}
	\part 求直线$ AB $的斜率;
	\part 设$ M $为曲线$ C $上一点,$ C $在$ M $处的切线与直线$ AB $平行,且$ AM\bm{\bot}BM $,求直线$ AB $的方程.
\end{parts}
%\qs 设$ O $为坐标原点,动点$ M $在椭圆$ C: \dfrac{x^2}{2}+y^2=1$上,过$ M $做$x$轴的垂线,垂足为$ N $,点$ P $满足$ \vv{NP}=\sqrt{2}\vv{NM} .$
%\begin{parts}
%	\part 求点$ P $的轨迹方程;
%	\part 设点$ Q $在直线$ x=3 $上,且$ \vv{OP}\cdot \vv{PQ} =1.$证明:过点$ P $且垂直于$ OQ $的直线$ l $过$ C $的左焦点$ F. $
%\end{parts}
\qs 在平面直角坐标系$xOy$中,曲线$ y=x^2+mx-2 $与$x$轴交于$ A,~B $两点,点$ C $的坐标为$ \left(0,1\right) $.当$ m $变化时,解答下列问题:
\begin{parts}
	\part 能否出现$ AC\bot BC $的情况?说明理由;
	\part 证明过$ A,~B,~C $三点的圆在$y$轴上截得的弦长为定值.
\end{parts}
\qs 设圆$ x^2+y^2+2x-15=0 $的圆心为$ A $,直线$ l $过点$ B(1,0) $且与$x$轴不重合,$ l $交圆$ A $与$ C~,D $两点,过$ B $作$ AC $的平行线交$ AD $于点$ E $.
\begin{parts}
	\part 证明$ \abs{EA}+\abs{EB} $为定值,并写出点$ E $的轨迹方程;
	\part 设点$ E $的轨迹为曲线$ C_1 $,直线$ l $交$ C_1 $于$ M,~N $两点,过$ B $且与$ l $垂直的直线与圆$ A $交于$ P,~Q $两点,求四边形$ MPNQ $面积的取值范围.
\end{parts}
\qs 已知椭圆$ E:\dfrac{x^2}{t}+\dfrac{y^2}{3}=1 $的焦点在$x$轴上,$ A $是$ E $的左顶点,斜率为$ k\left(k>0\right) $的直线交$ E $于$ A,~M $两点,点$ N $在$ E $上,$ MA\bot NA .$
\begin{parts}
	\part 当$ t=4,~\abs{AM}=\abs{AN} $时,求$ \triangle AMN $的面积;
	\part 当$ 2\abs{AM}=\abs{AN} $时,求$ k $的取值范围.
\end{parts}
\qs 已知抛物线$ C:y^2=2x $的焦点为$ F $,平行于$x$轴的两条直线$ l_1,~l_2 $分别交$ C $于$ A,~B $两点,交$ C $的准线于$P,~Q  $两点.
\begin{parts}
	\part 若$ F $在线段$AB$上,$ R $是$ PQ $的中点,证明$ AR\sslash FQ $;
	\part 若$ \triangle PQF $的面积是$ \triangle ABF $的面积的两倍,求$ AB $中点的轨迹方程.
\end{parts}
\qs 在平面直角坐标系$xOy$中,直线$ l:y=t\left(t\ne0\right) $交$y$轴于点$ M $,交抛物线$ C:  y^2=2px\left(p>0\right) $于点$ P $,$ M $关于$ P $的对称点为$ N $,连接$ ON $并延长交$ C $于点$ H $.
\begin{parts}
	\part 求$ \dfrac{\abs{OH}}{\abs{ON}} $;
	\part 除$ H $以外,直线$ MH $与$ C $是否有其它公共点?并说明理由. 		
\end{parts}
\qs 已知$ A $是椭圆$E$:$\dfrac{x^2}{4}+\dfrac{y^2}3=1$的左顶点,斜率为$ k\left(k>0\right) $的直线交$ E $于$ A,~M $两点,点$ N $在$ E $上,$ MA\bot NA $.
\begin{parts}
	\part 当$ \abs{AM}=\abs{AN} $时,求$ \triangle AMN $的面积;
	\part 当$ 2\abs{AM}=\abs{AN} $时,证明:$ \sqrt{3}<k<2. $
\end{parts}

\qs 在平面直角坐标系$xOy$中,曲线$ C:y=\dfrac{x^2}{4} $与直线$ l:y=kx+a(a>0) $交于$ M,~N $两点.
\begin{parts}
	\part 当$ k=0 $时,分别求$ C $在点$ M,~N $处的切线方程;
	\part $y$轴上是否存在点$ P $,使得当$ k $变动时,总有$ \angle OPM=\angle OPN $?说明理由.
\end{parts}
\qs 已知椭圆$ C:9x^2+y^2=m^2(m>0) $,直线$ l $不过原点$ O $且不平行与坐标轴,$ l $与$ C $有两个交点$ A,~B $,线段$ AB $的中点为$ M $.
\begin{parts}
	\part 证明:直线$ OM $的斜率与$ l $的斜率的乘积为定值;
	\part 若$ l $过点$ \left(\dfrac{m}{3},m\right) $,延长线段$ OM $与$ C $交于点$ P $,四边形$ OAPB $能否为平行四边形?若能,求此时$ l $的斜率;若不能,说明理由.
\end{parts}
\end{questions}





\section{导数}
\begin{questions}
	\qs 若$ x=-2 $是函数$f(x)=\left(x^2+ax-1\right)e^{x-1}$的极值点,则$f(x)$的极小值为\xx
	\onech{$ -1$}{$ -2e^{-3}$}{$ 5e^{-3}$}{$ 1$}
	\qs 已知函数$f(x)=x^2-2x+a(e^{x-1}+e^{-x+1})$有唯一零点,则$ a= $\xx
	\onech{$ -\dfrac{1}{2}$}{$ \dfrac{1}{3}$}{$ \dfrac{1}{2}$}{$ 1$}
	\qs 已知函数$f(x)=kx-\ln x$在区间$ \left(1,+\infty\right) $单调递增,则$ k $	的取值范围是\xx
	\onech{$ \left(-\infty,-2\right]$}{$ \left(-\infty,-1\right]$}{$ \left[2,+\infty\right)$}{$ \left[1,+\infty\right)$}
	\qs 设函数$f(x)=e^x\left(2x-1\right)-ax+a$,其中$ a<1 $,若存在唯一的整数$ x_0 $,使得$ f(x_0)<0 $,则$ a $的取值范围是\xx
	\onech{$ \left[-\dfrac{3}{2e},1\right)$}{$ \left[-\dfrac{3}{2e},\dfrac{3}{4}\right)$}{$\left[\dfrac{3}{2e},\dfrac{3}{4}\right) $}{$\left[\dfrac{3}{2e},1\right) $}
	\question
	设函数$f'(x)$是奇函数$f(x)~(x\in \mathbf{R})$的导函数,$f(-1)=0$,当$x>0$时,$xf'(x)-f(x)<0$,则使得$f(x)>0$成立的$x$的取值范围是\xx
	\twoch{$(-\infty,-1)\cup(0,1)$}{$(-1,0)\cup(1,+\infty)$}{$(-\infty,-1)\cup(-1,0)$}{$(0,1)\cup(1,+\infty)$}
	\qs 若直线$ y=kx+b $是曲线$ y=\ln x+2 $的切线,也是曲线$ y=\ln\left(x+1\right) $的切线,$b=$\tk.
	\qs 已知函数$f(x)=ae^{2x}+(a-2)e^x-x$.
	\begin{parts}
		\part 讨论$f(x)$的单调性;
		\part 若$f(x)$有两个零点,求$ a $的取值范围.
	\end{parts}
\qs 已知函数$f(x)=ax^2-ax-x\ln x $,且$f(x)\ge0$.
\begin{parts}
	\part 求$ a $;
	\part 证明:$f(x)$存在唯一的极大值点$ x_0 $,且$ e^{-2}<f(x_0)<2^{-3} $.
\end{parts}
\qs 已知函数$f(x)=x-1-a\ln x$.
\begin{parts}
	\part 若$f(x)\ge0$,求$a$的值;
	\part 设$ m $为整数,且对于任意正整数$ n $,$ \left(1+\dfrac{1}{2}\right)\left(1+\dfrac{1}{2^2}\right)\cdots\left(1+\dfrac{1}{2^n}\right)<m $,求$ m $的最小值.

\end{parts}
	\qs 已知函数$f(x)=e^x\left(e^x-a\right)-a^2x$.
\begin{parts}
	\part 讨论$f(x)$的单调性;
	\part 若$f(x)\ge0$,求$a$的取值范围.
\end{parts}
\qs 设函数$f(x)=(1-x^2)e^x$.
\begin{parts}
	\part 讨论$f(x)$的单调性;
	\part 当$ x\ge0 $时,$f(x)<ax+1$,求$ a $的取值范围.
\end{parts}
\qs 已知函数$f(x)=\ln x+ax^2+\left(2a+1\right)x$.
\begin{parts}
	\part 讨论$f(x)$的单调性;
	\part 当$ a<0 $时,证明$f(x)\le -\dfrac{3}{4a}-2$.
\end{parts}
\qs 
\begin{parts}
	\part 讨论函数$f(x)=\dfrac{x-2}{x+2}e^x$的单调性,并证明当$ x>0 $时,$ \left(x-2\right) e^x+x+2>0.$
	\part 当$ a\in\left[0,1\right) $时,函数$g(x)=\dfrac{e^x-ax-a}{x^2}\left(x>0\right)$有最小值,设$ g(x) $的最小值为$ h(a) $.求函数$ h(a) $的值域.
\end{parts}
	\qs 设函数$f(x)=a\cos 2x+\left(a-1\right)\left(\cos x+1\right),~$其中$ a>0 $,记$ \abs{f(x)} $的最大值为$A$.
	\begin{parts}
		\part 求$ f'(x) $;
		\part 求$ A $;
		\part 证明$ \abs{f'(x)}\le2A. $
	\end{parts}
\qs 已知函数$f(x)=\left(x+1\right)\ln x -a\left(x-1\right).$
\begin{parts}
	\part 当$ a=4 $时,求曲线$ y=f(x) $在$ \left(1,f(1)\right) $处的切线方程;
	\part 若当$ x\in\left(1,+\infty\right) $时,$ f(x)>0 $,求$ a $的取值范围.
\end{parts}
 \qs 设函数$f(x)=\ln x -x+1$.
 \begin{parts}
 	\part 讨论$f(x)$的单调性;
 	\part 证明当$ x\in\left(1,+\infty\right) $时,$ 1<\dfrac{x-1}{\ln x} <x$;
 	\part 设$ c>1 $,证明当$ x\in\left(0,1\right) $时,$ 1+\left(c-1\right)x>c^x $.
 \end{parts}
\qs 已知函数$f(x)=x^3+ax+\dfrac{1}{4},g(x)=-\ln x$
\begin{parts}
	\part 当$ a $为何值时,$x$轴为$y=f(x)$的切线;
	\part 用$ \min\{m,n\} $表示$ m,n $中的最小值,设函数$ h(x)=\min\left\{f(x),g(x)\right\}\left(x>0\right) $,讨论$ h(x) $零点的个数.
\end{parts}
\qs 设函数$f(x)=e^{mx}+x^2-mx$.
\begin{parts}
	\part 证明:$f(x)$在$ \left(-\infty,0\right) $单调递减,在$ \left(0,+\infty\right) $单调递增;
	\part 若对于任意$ x_1,x_2\in\left[-1,1\right] $,都有$ \abs{f(x_1)-f(x_2)}\le e-1 $,求$ m $的取值范围.
\end{parts}
\end{questions}





\section{排列组合二项式}
\begin{questions}
	\qs 安排$3$名志愿者完成$4$项工作,每人至少完成$1$项,每项工作由$1$人完成,则不同的安排方式共有\xx
	\onech{$ 12$种}{$18 $种}{$24 $种}{$36 $种}
	\qs $\left(1+\dfrac{1}{x^2}\right)\left(1+x\right)^6$展开式中$ x^2 $的系数为\xx
	\onech{$ 15$}{$ 20$}{$ 30$}{$ 35$}
	\qs $\left(x+y\right)\left(2x-y\right)^5$的展开式中$ x^3y^3 $的系数是\xx
	\onech{$ -80$}{$ -40$}{$ 40$}{$ 80$}
	\qs $ \left(x^2+x+y\right) ^5$的展开式中,$ x^5y^2 $的系数为\xx
	\onech{$ 10$}{$ 20$}{$ 30$}{$ 60$}
	\qs $ \left(a+x\right)\left(1+x\right)^4 $的展开式中$ x $的奇数次幂项的系数之和为$32$,则$ a=\tk $.
\end{questions}



\section{数列}
\begin{questions}
	\qs 记$S_n$ 为等差数列$\{a_n\}$的前$n$项和.若$ a_4+a_5=24,~S_6=48 $,则$\{a_n\}$的公差为\xx
	\onech{$ 1$}{$ 2$}{$ 4$}{$ 8$}
	\qs 等差数列$\{a_n\}$的前$ n $项和为$ S_n $,$ a_3=3,S_4=10, ~$则$ \sum\limits_{k=1}^{n}\dfrac{1}{S_k} =\tk$. 
	\qs 我国古代数学名著《算法统宗》中有如下问题:“远望巍巍塔七层,红光点点倍加增,共灯三百八十一,请问尖头几盏灯?”意思是:一座$7$层塔共挂了$381$盏灯,且相邻两层中的下一层灯数是上一层灯数的$2$倍,则塔的顶层共有灯\xx
	\onech{$ 1$盏}{$ 3$盏}{$5 $盏}{$9 $盏}
	\qs 等比数列$\{a_n\}$满足$a_1=3$,$a_1+a_3+a_5=21$,则$ a_3+a_5+a_7= $\xx
	\onech{21}{42}{63}{84}
	\qs 等比数列$\{a_n\}$满足$ a_1+a_3=10,~a_2+a_4=5,~ $则$ a_1a_2\cdots a_n $的最大值为\tk.
	\qs 设$S_n$是数列$\{a_n\}$的前$n$项和,且$a_1=-1$,$ a_{n+1=S_nS_{n+1}} $,则$ S_n=\tk $.
	\qs 设数列$\{a_n\}$满足$ a_1+3a_2+\cdots+\left(2n-1\right) a_n=2n.$
	\begin{parts}
		\part 求$\{a_n\}$的通项公式;
		\part 求数列$ \left\{\dfrac{a_n}{2n+1}\right\} $的前$ n $项和.
	\end{parts}
	\qs $S_n$为等差数列$\{a_n\}$的前n项和,且$ a_1=1,S_7=28,~ $记$ b_n=\left[\lg {a_n}\right] $,其中$ \left[x\right] $表示不超过$ x $的最大整数,如$ \left[0.9\right] =0,~\left[\lg {99}\right]=1.$
	\begin{parts}
		\part 求$ b_1,~b_{11},~b_{101} $;
		\part 求数列$ \left\{b_n\right\} $的前$ 1000 $项和.
	\end{parts}
	\qs 已知$\{a_n\}$是公差为$ 3 $的等差数列,数列$ \{b_n\} $满足$ b_1=1,~b_2=\dfrac{1}{3} ,~a_nb_{n+1}+b_{n+1}=nb_n$.
	\begin{parts}
		\part 求$ \{a_n\} $的通项公式;
		\part 求$\{b_n\}$的前$ n $项和.
	\end{parts}
\question
$S_n$为数列$\{a_n\}$的前$n$项和,已知$a_n>0$,$a_n^2+2a_n=4S_n+3$,其中$n\in \mathbf{N}^*$.
\begin{parts}
	\part 求$\{a_n\}$的通项公式;
	\part 设$b_n=\dfrac{1}{a_na_{n+1}}$,求数列$\{b_n\}$的前$n$项和.
\end{parts}

\end{questions}





\section{不等式}
\begin{questions}
	\qs 设$x,~y$满足约束条件$\begin{dcases}
	x+2y\le1\\
	2x+y\ge-1\\
	x-y\le0
	\end{dcases}$,则$ z=3x-2y $的最小值为\tk.
	\qs 设$x,~y$满足约束条件$\begin{dcases}
	x-1\ge0,\\
	x-y\le 0,\\
	x+y-4\le 0
	\end{dcases}$则$ \dfrac{y}{x} $的最大值为\tk.
\end{questions}




\section{逻辑与命题}
\begin{questions}
	\qs 设有下面四个命题:\\
	$p_1:$若复数$ z $满足$\dfrac1z\inR$,则$ z\inR $;\\
	$p_2:$若复数$z$满足$ z^2\inR $,则$ z\inR $;\\
	$p_3:$若复数$z_1,~z_2$满足$ z_1z_2\inR $,则$ z_1=\overline{z_2} $;\\
	$p_4:$若复数$ z\inR $,则$ \overline{z}\inR $.\\
	其中真命题为\xx
	\onech{$ p_1,p_3$}{$ p_1,p_4$}{$ p_2,p_3$}{$ p_2,p_4$}
	\qs 函数$f(x)$在$ x=x_0 $处的导数存在,若$ p:f'(x_0)=0,~q:x=x_0 $是$f(x)$的极值点,则\xx
	\fourch{$ p $是$ q $的充分必要条件}{$ p$是$ q $的充分条件,但不是$ q $的必要条件}{$ p$是$ q $的必要条件,但不是$ q $的充分条件}{$ p$既不是$ q $的充分条件,也不是$ q $的必要条件}
	\qs 设命题$ P:\exists n\inN ,n^2>2^n$,则$ \neg P $为\xx
	\fourch{$ \forall n\inN,~n^2>2^n$}{$  \exists n\inN,~n^2\le2^n$}{$ \forall n\inN,~n^2\le2^n $}{$ \exists n\inN,~n^2=2^n$}
\end{questions}



\end{document}