\documentclass{BHCexam}	
\begin{document}
\biaoti{导数选填}
\fubiaoti{}
\maketitle
\begin{questions}
\question
若$f(x)$的导数为$f'(x)$且满足$f'(x)<f(x)$,则$f(3)$与$e^3f(0)$的大小关系是\xx
\onech{$f(3)<e^3f(0)$}{$f(3)=e^3f(0)$}{$f(3)>e^3f(0)$}{不能确定}
\question
设$f(x)$,$g(x)$是$\mathbf{R}$上的可导函数,$f'(x)$,$g'(x)$分别是$f(x)$,$g(x)$的导函数,且$f'(x)g(x)-f(x)g'(x)<0$,$g(x)>0$对$x\in\mathbf{R}$恒成立,则当$a<b$时,有\xx
\twoch{$f(a)g(b)>f(b)g(a)$}{$f(b)g(b)<f(a)g(a)$}{$f(b)g(b)>f(a)g(a)$}{$f(a)g(b)<f(b)g(a)$}
\qs 若$ 0<x_1<x_2<1 $,则\xx
\twoch{$ \mathrm{e}^{x_2}-\mathrm{e}^{x_1}>\ln{x_2}-\ln{x_1}$}{$\mathrm{e}^{x_2}-\mathrm{e}^{x_1}<\ln{x_2}-\ln{x_1} $}{$ x_2\mathrm{e}^{x_1}>x_1\mathrm{e}^{x_2}$}{$x_2\mathrm{e}^{x_1}<x_1\mathrm{e}^{x_2} $}
\question
设函数$f(x)=\sqrt{3}\sin\dfrac{\pi x}{m}$.若存在$f(x)$的极值点$x_0$满足$x_0^2+\left[f(x_0)\right]^2<m^2$,则$m$的取值范围是\xx
\twoch{$(-\infty,-6)\cup(6,+\infty)$}{$(-\infty,-4)\cup(4,+\infty)$}{$(-\infty,-2)\cup(2,+\infty)$}{$(-\infty,-1)\cup(1,+\infty)$}
\question
已知函数$f(x)=ax^3-3x^2+1$,若$f(x)$存在唯一的零点$x_0$,且$x_0>0$,则$a$的取值范围是\\
\mbox{\hspace{2em}}\hfill\xx
\onech{$\left(2,+\infty \right)$}{$\left(-\infty,-2\right)$}{$\left(1,+\infty\right)$}{$\left(-\infty,-1\right)$}
\question
已知函数$f(x)=x^3+ax^2+bx+c$,给出下来结论:\\
\ding{172}~$\exists x_0 \in \mathbf{R}$,$f(x_0)=0$;\\
\ding{173}~函数$f(x)$的图像是中心对称图形;\\
\ding{174}~若$x_0$是$f(x)$的极小值点,则$f(x)$在$(-\infty,x_0)$上单调递减;\\
\ding{175}~若$x_0$是$f(x)$的极值点,则$f'(x_0)=0$\\
上述结论错误的是\xx
\onech{\ding{172}\ding{174}}{\ding{173}\ding{174}}{\ding{173}\ding{175}}{\ding{174}}
\question
设函数$f'(x)$是奇函数$f(x)~(x\in \mathbf{R})$的导函数,$f(-1)=0$,当$x>0$时,$xf'(x)-f(x)<0$,则使得$f(x)>0$成立的$x$的取值范围是\xx
\twoch{$(-\infty,-1)\cup(0,1)$}{$(-1,0)\cup(1,+\infty)$}{$(-\infty,-1)\cup(-1,0)$}{$(0,1)\cup(1,+\infty)$}
%\question
%设函数$f(x)=e^x(2x-1)-ax+a$,其中$a<1$,若存在唯一的整数$x_0$使得$f(x_0)<0$,则$a$的取值范围是\xx
%\twoch{$\Big[-\dfrac{3}{2e},1\Big)$}{$\Big[-\dfrac{3}{2e},\dfrac{3}{4}\Big)$}{$\Big[\dfrac{3}{2e},\dfrac{3}{4}\Big)$}{$\Big[\dfrac{3}{2e},1\Big)$}
\qs 设函数$f(x)$的定义域为$ \mathbf{R} ,~x_0~(x_0\ne 0)$是$f(x)$的极大值点,以下结论正确的是\xx
\twoch{$ \forall x\in \mathbf{R},f(x)\le f(x_0) $}{$ -x_0\text{是}f(-x)\text{的极小值点} $}{$ -x_0\text{是}-f(x)\text{的极小值点} $}{$ -x_0 \text{是}-f(-x)\text{的极小值点} $}
\question
已知函数$f(x)=\begin{dcases}
-x^2+2x,&x\le0,\\
\ln(x+1),&x>0.
\end{dcases}$若$\abs{f(x)}\ge ax$,则$a$的取值范围是\xx
\onech{$\left(-\infty,0\right]$}{$\left(-\infty,1\right]$}{$\left[-2,-1\right]$}{$\left[-2,0\right]$}
\question
若函数$f(x)=x-\dfrac{1}{3}\sin 2x+a\sin x$在$(-\infty,+\infty)$单调递增,则$a$的取值范围是\xx
\onech{$\left[-1,1\right]$}{$\left[-1,\dfrac{1}{3}\right]$}{$\left[-\dfrac{1}{3},\dfrac{1}{3}\right]$}{$\left[-1,-\dfrac{1}{3}\right]$}
\qs 已知函数$f(x)=x^3+ax^2+bx+c$有两个极值点$x_1,x_2$,若$f(x_1)=x_1<x_2$,则关于$x$的方程$3(f(x))^2+2af(x)+b=0$的不同实数根的个数为\xx
\onech{3}{4}{5}{6}
\qs 已知函数$f(x)=e^x+ax-2$,其中$ a\inR $.若对于任意的$ x_1,x_2\in\left[1,+\infty\right) $,且$ x_1<x_2 $,都有$ x_2f(x_1)-x_1f(x_2)<a(x_1-x_2) $,则$ a $的取值范围是\xx
\twoch{$ \left[1,+\infty\right)$}{$\left[2,+\infty\right) $}{$\left(-\infty,1\right] $}{$\left(-\infty,2\right]  $}
\question
设函数
$f(x)=\begin{dcases}
x^3-3x &x\le a,\\
-2x &x>a.
\end{dcases}$\\
\ding{172}若$a=0$,则$f(x)$的最大值为\tk;\\
\ding{173}若$f(x)$无最大值,则实数$a$的取值范围是\tk.
\question
已知$f(x)$为偶函数,当$x<0$时,$f(x)=\ln \left(-x\right)+3x$,则曲线$y=f(x)$在点$(1,-3)$处的切线方程是\tk.
\qs 已知曲线$y=x+\ln x$在点$(1,1)$处的切线与曲线$y=ax^2+(a+2)x+1$相切,则$a=$\tk.
\end{questions}
\end{document}