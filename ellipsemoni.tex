\documentclass[marginline,noindent,answers,adobefonts]{BHCexam}	

\newcommand{\xl}[2]{\vv{#1}\bm\cdot\vv{#2}}
\begin{document}
\begin{questions}
\qs 已知椭圆$C$:$\dfrac{x^2}{4}+\dfrac{y^2}{3}=1~$的左,右焦点分别为$ F_1,F_2 ,$,椭圆$C$上的点$ A $满足$ AF_2\bot F_1F_2. $若点$ P $是椭圆$C$上的动点,则$ \vv{F_1P}\cdot\vv{F_2A} $的最大值为\xx
\onech{$ \dfrac{\sqrt{3}}{2} $}{$ \dfrac{3\sqrt{3}}{2} $}{$ \dfrac{9}{4} $}{$ \dfrac{15}{4} $}
\qs“$ m<8 $”是“方程$\dfrac{x^2}{m-10}-\dfrac{y^2}{m-8}=1$表示双曲线”的\xx
 \twoch{充分而不必要条件}{必要而不充分条件}{充分必要条件}{既不充分也不必要条件}
\qs 设抛物线$ C:y^2=4x $的焦点为$ F,~ $$ M $为抛物线$ C $上一点,$ N(2,2) ,~$则$ \left|MF \right|+\left| MN\right|$的取值范围是\tk.
\qs 已知函数$ f(x) =e^{-\left|x\right|}+\cos \pi x$~,~给出下列命题:\\
\ding{192} $f(x)$的最大值为2;\\
\ding{193} $f(x)$在$ (-10,10) $内的零点之和为0;\\
\ding{194} $f(x)$的任何一个极大值都大于1.\\
其中,所有正确命题的序号是\tk.
\qs 设抛物线$ y^2=8x $的焦点为$ F $,准线为$ l $,$ P $为抛物线上一点,$ PA\bm{\bot} l $,$ A $为垂足,若直线$ AF $的斜率为$ -\sqrt{3} $,则$ \left|PF\right|= $\xx
\onech{$4\sqrt{3} $}{$ 6$}{$ 8$}{$ 16$}

\newpage
\qs 已知椭圆$C$:$\dfrac{x^2}{a^2}+\dfrac{y^2}{b^2}=1~(a>b>0)$的离心率为$ \dfrac{\sqrt{3}}{2} $,椭圆$C$与$ y $轴交于$ A,~B $两点,且$ \left|AB\right|=2 $.
\begin{parts}
\part 求椭圆$C$的方程;
\part 设点$ P $是椭圆$C$上的一个动点,且点$ P $在$ y $轴右侧,直线$ PA,~PB $与直线$ x=4 $分别交于$ M,~N $两点,若以$ MN $为直径的圆与$ x $轴交于两点$ E,~F $,求点$ P $横坐标的取值范围及$ \left|EF\right| $的最大值.
\end{parts}
\kongbai
\qs 已知椭圆$C$:$\dfrac{x^2}{a^2}+\dfrac{y^2}{b^2}=1~(a>b>0)$的离心率为$ \dfrac{\sqrt{3}}{2} $,椭圆$C$与$ y $轴交于$ A,~B $两点,且$ \left|AB\right|=2$.
\begin{parts}
\part 求椭圆$C$的方程; 
\part 设点$ P $是椭圆$C$上的一个动点,且直线$ PA, ~PB $与直线$ x=4 $分别交于$ M,~N $两点,是否存在点$ P $使得以$ MN $为直径的圆经过定点$ (2,0) ?$若存在,求出$ P $点坐标;若不存在,说明理由.
\end{parts}
\kongbai
\qs 已知点$ A(x_1,y_1),~D(x_2,y_2)~(\text{其中}x_1<x_2)$是曲线$ y^2=4x~(y\ge0) $上的两点,$ A,~D $两点在$ x $轴上的射影分别为$ B,~C $,且$ |BC|=2. $
\begin{parts}
\part 当点$ B $的坐标为$ (1,0) $时,求直线$ AD $的斜率;
\part 记$ \triangle OAD $的面积为$ S_1 $,梯形$ ABCD $的面积为$ S_2 $,求证:$ \dfrac{S_1}{S_2}<\dfrac{1}{4} .$
\end{parts}
\kongbai
\qs 已知椭圆$C$:$ mx^2+3my^2=1~(m>0) $的长轴长为$ 2\sqrt{6},~ O$为坐标原点.
\begin{parts}
\part 求椭圆$C$的方程和离心率;
\part 设点$ A(3,0) $,动点$ B $在$y$轴上,动点$ P $在椭圆$C$上,且$ P $在$y$轴右侧,若$\left|BA \right|=\left|BP \right|$,求四边形$ OPAB $面积的最小值.
\end{parts}
\kongbai
\qs 已知椭圆$C$:$ mx^2+3my^2=1~(m>0) $的长轴长为$ 2\sqrt{6},~ O$为坐标原点.
\begin{parts}
\part 求椭圆$C$的方程和离心率;
\part 设动直线$ l $与$y$轴相交于点$ B $,~点$ A(3,0) $关于直线$ l $的对称点$ P $在椭圆$C$上,求$ |OB| $的最小值.
\end{parts}
\kongbai
\qs 设$ F_1,~F_2 $分别为椭圆$E$:$\dfrac{x^2}{a^2}+\dfrac{y^2}{b^2}=1~(a>b>0)$的左右焦点,点$ A $为椭圆$E$的左顶点,点$B$为椭圆$E$的上顶点,且$ \left|AB\right|=2 $.
\begin{parts}
\part 若椭圆$ E $的离心率为$ \dfrac{\sqrt{6}}{3},~ $求椭圆$E$的方程;
\part 设$ P $为椭圆$ E $上一点,且在第一象限内,直线$ F_2P $与$y$轴相交于点$ Q $,若以$ PQ $为直径的圆经过点$ F_1 $,证明$ \left|OP\right|>\sqrt{2}. $
\end{parts}

\kongbai
\qs 已知椭圆$C$:$\dfrac{x^2}{a^2}+\dfrac{y^2}{b^2}=1~(a>b>0)$的两个焦点和短轴的两个顶点构成的四边形是一个正方形,且其周长为$ 4\sqrt{2} .$
\begin{parts}
\part 求椭圆$C$的方程;
\part 设过点$ B(0,m)~(m>0) $的直线$ l $与椭圆$C$相交于$ E,~F $两点,点$ B $关于原点的对称点为$ D $,若点$ D $总在以线段$ EF $为直径的圆内,求$ m $的取值范围.
\end{parts}
\kongbai
\qs 已知椭圆$M$:$\dfrac{x^2}{a^2}+\dfrac{y^2}{b^2}=1~(a>b>0)$过点$A(0,-1)  $,且离心率$ e=\dfrac{\sqrt{3}}{2} .$
\begin{parts}
\part 求椭圆$M$的方程;
\part 若椭圆$M$上存在点$ B,C $关于直线$ y=kx-1 $对称,求$ k $的所有取值构成的集合$ S $,并证明对于$ \forall k\in S $,$BC  $的中点恒在一条定直线上.
\end{parts}
\kongbai
\qs 已知椭圆$W$:$\dfrac{x^2}{2}+y^2=1~,$直线$ l $与$ W $相交于$ M,~N $两点,$ l $与$x$轴、$y$轴分别相交于$ C,~D $两点,$ O $为坐标原点.
\begin{parts}
\part 若直线$ l $的方程为$ x+2y-1=0,~ $求$ \triangle OCD $外接圆的方程;
\part 判断是否存在直线$ l,~ $使得$ C,~D $是线段$ MN $的两个三等分点,若存在,求出直线$ l $的方程;若不存在,说明理由.
\end{parts}
\kongbai
\vspace{-1em}
\qs 已知椭圆$W$:$\dfrac{x^2}{a^2}+\dfrac{y^2}{b^2}=1~(a>b>0)$的焦距为2,过右焦点和短轴的一个端点的直线的斜率为$ -1 $,$ O $为坐标原点.
\begin{parts}
\part 求椭圆$ W $的方程;
\part 设斜率为$ k $的直线$ l $与$ W $相交于$ A,B $两点,记$ \triangle AOB $面积的最大值为$ S_k $,证明: $ S_1=S_2. $
\end{parts}
\kongbai
\qs 设$ A,B $是椭圆$ W:\dfrac{x^2}{4}+\dfrac{y^2}{3}=1 $上不关于坐标轴对称的两个点,直线$ AB $交$x$轴于点$ M ~$$\left(\text{与点} A,B \text{不重合}\right)$,$ O $为坐标原点.
\begin{parts}
\part 如果点$ M $是椭圆$ W $的右焦点,线段$ MB $的中点在$y$轴上,求直线$ AB $的方程;
\part 设$ N $为$x$轴上一点,且$\xl{OM}{ON}=4$,直线$ AN $和椭圆$ W $的另外一个交点为$ C $,证明:点$B  $与点$ C $关于$x$轴对称.
\end{parts}
\kongbai
\qs 已知$ A,~B $是椭圆$ C:2x^2+3y^2=9 $上两点,点$ M $的坐标为$ (1,0) .$
\begin{parts}
\part 当$ A,~B $两点关于$x$轴对称,且$ \triangle MAB $为等边三角形时,求$ AB $的长;
\part 当$ A,~B $两点不关于$x$轴对称时,证明:$ \triangle MAB $不可能为等边三角形.
\end{parts}
\kongbai
\qs 已知椭圆$G$的离心率为$ \dfrac{ \sqrt{2}}{ 2}$,其短轴两端点为$ A(0,1),~B(0,-1). $
\begin{parts}
\part 求椭圆$ G $的方程;
\part 若$ C,~D $是椭圆$ G $上关于$y$轴对称的两个不同点,直线$ AC,~BD $与$x$轴交于点$ M ,~N,~$判断以$ MN $为直径的圆是否经过点$ A $.并说明理由.
\end{parts}
\kongbai
\qs 已知椭圆$C$:$\dfrac{x^2}{a^2}+\dfrac{y^2}{b^2}=1~(a>b>0)$的离心率为$ \dfrac{\sqrt{2}}{2},~ $右焦点为$ F,~ $点$ P(0,1) $在椭圆$ C $上.
\begin{parts}
\part 求椭圆$C$的方程;
\part 过点$ F $的直线交椭圆$C$于$ M,~N $两点,交直线$ x=2 $于点$ P ,$设$ \vv{PM}=\lambda\vv{MF},~ \vv{PN}=\mu \vv{NF}$,~求证:$ \lambda+\mu $为定值.
\end{parts}
\kongbai
\qs 已知$ A(0,2),~B(3,1) $是椭圆$G$:$\dfrac{x^2}{a^2}+\dfrac{y^2}{b^2}=1~(a>b>0)$上的两点.
\begin{parts}
\part 求椭圆$G$的离心率;
\part 已知直线$ l $过点$ B $,且与椭圆$ G $交于另一点$ C $(不同于点$ A $),若以$ BC $为直径的圆经过点$ A $,求直线$ l $的方程.
\end{parts}
\kongbai
\qs 已知椭圆$C$:$\dfrac{x^2}{a^2}+\dfrac{y^2}{b^2}=1~(a>b>0)$上的点到它的两个焦点的距离之和为 $ 4 $,以椭圆$ C $的短轴为直径的圆$ O $经过这两个焦点,点$ A,~B $分别是椭圆$ C $的左、右顶点.
\begin{parts}
\part 求圆$ O $和椭圆$ C $的方程;
\part 已知$ P,~ Q$分别是椭圆$C$和圆$ O $上的动点($ P,~Q $位于$y$轴两侧),且直线$PQ$与$x$轴平行,直线$ AP,~BP $分别于$y$轴交于点$ M,~N $.求证:$ \angle MQN $为定值.
\end{parts}
\kongbai
\qs 已知椭圆的中心为坐标原点$ O $,焦点在$x$轴上,离心率$ e=\dfrac{\sqrt{6}}{3},~ $斜率为1且过椭圆右焦点$ F $的直线交椭圆于$ A,~B $两点,$ M $为椭圆上任一点,且$ \vv{OM}=\lambda\vv{OA}+\mu \vv{OB},~ $证明$ \lambda^2+\mu^2 $为定值.
\kongbai
\qs 已知椭圆$C$:$\dfrac{x^2}{9}+\dfrac{y^2}{5}=1~$的左右顶点分别为$ A,~B,~$右焦点为$ F ,~$设过点$ T(9,m) $的直线$ TA,~TB $与椭圆分别交于点$ M(x_1,~y_1),~N(x_2,~y_2) ,~$其中$ m>0,~y_1>0,~y_2<0.~ $求证:直线$ MN  $必过$x$轴上的一定点.
\newpage
\qs 已知椭圆$C$:$\dfrac{x^2}{a^2}+y^2=1~(a>1)$,离心率$ e=\dfrac{\sqrt{6}}{3}. $直线$ l:x=my+1 $与$x$轴交于点$ A $,与椭圆$ C $交于点$ E,~F $两点.自点$ E,~F $分别向直线$ x=3 $做垂线,垂足分别为$ E_1,~F_1. $
\begin{parts}
\part 求椭圆$C$的方程及焦点坐标;
\part 记$ \triangle AEE_1,~\triangle AE_1F_1,~ \triangle AFF_1$的面积分别为$ S_1,~S_2,~S_3,~ $试证明$ \dfrac{S_1S_3}{S_2^2} $为定值.
\end{parts}
\kongbai
\qs 已知点$ P $是椭圆$C$:$\dfrac{x^2}{a^2}+\dfrac{y^2}{b^2}=1~(a>b>0)$上一点,点$ P $到椭圆$C$的两个焦点的距离之和为$ 2\sqrt{2} $.
\begin{parts}
\part 求椭圆$C$的方程;
\part 设$ A,~B $是椭圆$C$上异于点$P$的两点,直线$ PA $与直线$ x=4 $交于点$ M $,~是否存在点$ A,~ $使得$ S_{\triangle ABP} =\dfrac{1}{2}S_{\triangle ABM}$?若存在,求出点$ A $的坐标;若不存在,说明理由.
\end{parts}
\kongbai
\qs 已知椭圆$C$:$\dfrac{x^2}{a^2}+\dfrac{y^2}{b^2}=1~(a>b>0)$的离心率为$ \dfrac{\sqrt{3}}{2},~ $短半轴长为$ 1. $
\begin{parts}
\part 求椭圆$ G $的方程;
\part 设椭圆$ G $的短轴端点分别为$ A,~B $,点$ P $是椭圆$ G $上异于点$ A,~B $的一动点,直线$ PA,~PB $分别与直线$ x=4 $交于$ M,~N $两点,以线段$ MN $为直径作圆$ C $.\\
\ding{192} 当点$ P $在$y$轴的左侧时,求圆$ C $半径的最小值;\\
\ding{193} 问:是否存在一个圆心在$x$轴上的定圆与圆$ C $相切?若存在,指出该定圆的圆心和半径,并证明你的结论;若不存在,说明理由.
\end{parts}
\end{questions}
\end{document}