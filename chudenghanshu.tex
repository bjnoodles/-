\documentclass{BHCexam}
\usepackage[hidelinks]{hyperref}
\begin{document}
\biaoti{初等函数}
\fubiaoti{}
\maketitle
%\tableofcontents
\begin{questions}
\qs 已知函数$f(x)=3^x-\left(\dfrac{1}{3}\right)^x$,则$f(x)$\xx
\twoch{是偶函数,且在$\mathbf{R}$上是增函数}{是奇函数,且在$\mathbf{R}$上是增函数}{是偶函数,且在$\mathbf{R}$上是减函数}{是奇函数,且在$\mathbf{R}$上是减函数}
\qs 根据有关资料,围棋状态空间复杂度的上限$ M $约为$ 3^{361} $,而可观测宇宙中可观测物质的总数$ N $约为$ 10^{80} $.则下列各数中与$ \dfrac{M}{N} $最接近的是\xx\\
(参考数据:$ \lg 3\approx 0.48 $)\\
\onech{$ 10^{33}$}{$ 10^{53}$}{$ 10^{73}$}{$ 10^{93}$}
\qs 若$ \ln a>0,\log_32^b<-1,c^2\le1, $那么$ a,b,c $中最大的一个是\tk.
\end{questions}
\end{document}