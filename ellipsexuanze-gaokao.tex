\documentclass[marginline,noindent,answers,adobefonts]{BHCexam}

%\newcommand{\ty}{\dfrac{x^2}{a^2}+\dfrac{y^2}{b^2}=1~(a>b>0)}
\begin{document}

\biaoti{圆锥曲线选填}
\fubiaoti{}
\maketitle
\section{直线与圆}
\begin{questions}
\qs 已知点$ M(a,b) $在圆$ O:x^2+y^2=1 $外,则直线$ ax+by=1 $与圆$ O $的位置关系是\xx
\onech{相切}{相交}{相离}{不确定} 
\qs 已知圆$ C:x^2+y^2-4x=0 ,l$过点$ P(3,0) $的直线,则\xx
\fourch{$ l$与$ C $相交}{$ l$与$ C $相切}{$ l$与$ C $相离}{以上三个选项均有可能}
\end{questions}
\section{圆锥曲线}
\begin{questions}
\question
已知$O$为坐标原点,$F$是椭圆$C$:$\dfrac{x^2}{a^2}+\dfrac{y^2}{b^2}=1~(a>b>0)$的左焦点,$A$,$B$分别是$C$的左、右顶点.~$P$为$C$上一点,且$PF\bot x$轴,且过点$A$的直线$l$与线段$PF$交于点$M$,与$y$轴交于点$E$,若直线$BM$经过$OE$的中点,则$C$的离心率为\xx 
\onech{$\dfrac13$}{$\dfrac12$}{$\dfrac23$}{$\dfrac34$}
\question
已知方程$\dfrac{x^2}{m^2+n}-\dfrac{y^2}{3m^2-n}=1$表示双曲线,且该双曲线两焦点间的距离为4,则$n$的取值范围是\xx
\twoch{$(-1,3)$}{$(-1,\sqrt{3})$}{$(0,3)$}{$(0,\sqrt{3})$}
\question
已知$M(x_0,y_0)$是双曲线$C$:$\dfrac{x^2}{2}-y^2=1$上的一点,$F_1,F_2$是$C$的两个焦点.~~若$\vv{MF_1}\cdot\vv{MF_2}<0$,则$y_0$的取值范围是\xx
\twoch{$\Big(-\dfrac{\sqrt{3}}{3},\dfrac{\sqrt{3}}{3}\Big)$}{$\Big(-\dfrac{\sqrt{3}}{6},\dfrac{\sqrt{3}}{6}\Big)$}{$\Big(-\dfrac{2\sqrt{2}}{3},\dfrac{2\sqrt{2}}{3}\Big)$}{$\Big(-\dfrac{2\sqrt{3}}{3},\dfrac{2\sqrt{3}}{3}\Big)$}
\question
已知椭圆$E$:$\dfrac{x^2}{a^2}+\dfrac{y^2}{b^2}=1(a>b>0)$的右焦点为$F(3,0)$,过点$F$的直线交$E$于$A,B$两点,若$AB$的中点坐标为$(1,-1)$,则$E$的方程为\xx
\twoch{$\dfrac{x^2}{45}+\dfrac{y^2}{36}=1$}{$\dfrac{x^2}{36}+\dfrac{y^2}{27}=1$}{$\dfrac{x^2}{27}+\dfrac{y^2}{18}=1$}{$\dfrac{x^2}{18}+\dfrac{y^2}{9}=1$}
\qs
已知椭圆$C$:$\dfrac{x^2}{a^2}+\dfrac{y^2}{b^2}=1~(a>b>0)$的左,右焦点分别为$F_1$,$F_2$,离心率为$\dfrac{\sqrt{3}}{3}$,过$F_2$的直线$l$交$C$于$A,B$两点,若$\triangle AF_1B$的周长为$4\sqrt{3}$,则$C$的方程为\xx
\twoch{$\dfrac{x^2}{3}+\dfrac{y^2}{2}=1$}{$\dfrac{x^2}{3}+y^2=1$}{$\dfrac{x^2}{12}+\dfrac{y^2}{8}=1$}{$\dfrac{x^2}{12}+\dfrac{y^2}{4}=1$}
\qs
已知椭圆$C$:$\dfrac{x^2}{4}+\dfrac{y^2}{3}=1$的左,右焦点分别为$F_1,F_2$,椭圆$C$上的点$A$满足$AF_2\bot F_1F_2$,若点$P$是椭圆$C$上的动点,则$\vv{F_1P}\cdot\vv{F_2A}$的最大值为\xx  
\onech{$\dfrac{\sqrt{3}}{2}$}{$\dfrac{3\sqrt{3}}{2}$}{$\dfrac{9}{4}$}{$\dfrac{15}{4}$}
\qs 已知动点$ P(x,y) $在椭圆$ C:\dfrac{x^2}{25}+\dfrac{y^2}{16}=1 $上,$ F $为椭圆$ C $的右焦点,若点$ M $满足$ |\vv{MF}|=1 $且$\vv{MP}\cdot\vv{MF}=0  $,则$ |\vv{PM}| $的最小值为\xx
\onech{$ \sqrt{3} $}{3}{$ \dfrac{12}{5} $}{1}
\qs 设点$ M(x_0,1),~ $若在圆$ O: x^2+y^2=1 $上存在点$ N $,使得$ \angle OMN=45^{\circ} $,则$ x_0 $的取值范围是\xx
\onech{$ \left[-1,1\right]$}{$ \left[-\dfrac{1}{2},\dfrac{1}{2}\right]$}{$\left[-\sqrt{2},\sqrt{2}\right] $}{$ \left[-\dfrac{\sqrt{2}}{2},\dfrac{\sqrt{2}}{2}\right]$}
\qs 设$ F_1,F_2   $是椭圆$E:\dfrac{x^2}{a^2}+\dfrac{y^2}{b^2}=1~(a>b>0)$的左,右焦点,$ P $为直线$ x=\dfrac{3a}{2} $上一点,$ \triangle F_1PF_2 $是底角为$ 30^{\circ} $的等腰三角形,则$ E $的离心率为\xx
\onech{$\dfrac{1}{2}$}{$\dfrac{2}{3}$}{$\dfrac{3}{4}$}{$\dfrac{4}{5}$}
\qs 已知$ F $为双曲线$ C:x^2-my^2=3m~(m>0)$的一个焦点,则点$ F $到$ C $的一条渐近线的距离为\xx
\onech{$ \sqrt{3}$}{$ 3$}{$ \sqrt{3}m$}{$ 3m$}

\qs 设抛物线$ C:y^2=2px~(p>0) $的焦点为$ F $,点$ M $在$ C $上,$ \abs{MF}=5 $,若以$ MF $为直径的圆过点$ \left(0,2\right) $,则$ C $的方程为\xx
\twoch{$ y^2=4x$或$ y^2=8x $}{$ y^2=2x$或$ y^2=8x $}{$ y^2=4x$或$ y^2=16x $}{$ y^2=2x$或$ y^2=16x $}
\qs 设双曲线$ C $的中心为点$ O $,若有且只有一对相交于点$ O $、所成角为$ 60^{\circ} $的直线$ A_1B_1 $和$ A_2B_2 $,使得$ \abs{A_1B_1}=\abs{A_2B_2} $,其中$ A_1,~B_1 $和$ A_2,~B_2 $分别是这对直线与双曲线$ C $的交点,则该双曲线的离心率的取值范围是\xx
\twoch{$ \left(\dfrac{2\sqrt{3}}{3},2\right]$}{$ \left[\dfrac{2\sqrt{3}}{3},2\right)$}{$\left(\dfrac{2\sqrt{3}}{3},+\infty\right) $}{$ \left[\dfrac{2\sqrt{3}}{3},+\infty\right)$}
\qs 已知圆$ C:(x-3)^2+(y-4)^2=1 $和两点$ A(-m,0),B(m,0)~(m>0) $,若圆上存在点$ P $,使得$ \angle APB=90^{\circ} $,则$ m $的最大值为\xx
\onech{7}{6}{5}{4}

\qs 在平面直角坐标系$xOy$中,直线$ l $过抛物线$y^2=4x$的焦点$ F $,且与抛物线相交于$ A,B $两点,其中点$ A $在$x$轴上方.若直线$ l $的倾斜角为$ 60^{\circ} $,则$ \triangle OAF $的面积为\tk.
\qs 曲线$ C $是平面内与两个定点$ F_1(-1,0) $和$ F_2(1,0) $的距离的积等于常数$ a^2 (a>1)$的点的轨迹,给出以下三个结论:\\
\ding{192}曲线$ C $过坐标原点;\\
\ding{193}曲线$ C $关于原点对称;\\
\ding{194}若点$ P $在曲线$ C $上,则$ \triangle F_1PF_2 $的面积大于$ \dfrac{1}{2}a^2 $.\\
其中,所有正确的结论的序号是\tk.
\qs 已知双曲线$\dfrac{x^2}{a^2}-\dfrac{y^2}{b^2}=1$的离心率为$ 2 $,焦点与椭圆$ \dfrac{x^2}{25}+\dfrac{y^2}{9}=1 $的焦点相同,那么双曲线的焦点坐标为\tk;渐近线方程为\tk.
\qs 设双曲线$ C $经过点$ \left(2,2\right) $,且与$ \dfrac{y^2}{4}-x^2=1 $具有相同渐近线,则$ C $的方程为\tk;渐近线方程为\tk. 
\qs 已知动点$ P $到定点$ A(-2,0) $与点$ B$的斜率之积为$ -\dfrac{1}{4} $,点$ P $的轨迹方程为\tk.
\qs 双曲线$\dfrac{x^2}{a^2}-\dfrac{y^2}{b^2}=1~(a>0,b>0)$的渐近线为正方形$ OABC $的边$ OA,OC $所在的直线,点$ B $为该双曲线的焦点,若正方形$ OABC $的边长为$ 2 $,则$ a= $\tk.
\qs 若双曲线$ M $上存在四个点$ A,B,C,D $,使得四边形$ ABCD $是正方形,则双曲线$ M $的离心率的取值范围是\tk.
\qs 设$ F_1,F_2 $分别为椭圆$ \dfrac{x^2}{3}+y^2=1 $的左、右焦点,点$ A,~B $在椭圆上,若$ \vv{F_1A}=5\vv{F_2B} $,则点$ A $的坐标是\tk.
\qs 设直线$ x-3y+m=0 ~(m\ne0)$与$ \dfrac{x^2}{a^2}-\dfrac{y^2}{b^2}=1~(a>0,b>0)$的两条渐近线分别交于$ A,B .$若点$ P(m,0) $满足$ \abs{PA}=\abs{PB} $,则该双曲线的离心率是\tk.
\end{questions}
\end{document}