\documentclass{BHCexam}	
%\usepackage{titlesec}%自定义title格式
\usepackage{enumerate}

\begin{document}
\newcounter{example}%[section]
\renewcommand{\theexample}{\arabic{example}}
\newenvironment{example}[1][]{\par \hspace{-3.5em}\refstepcounter{example}\textbf{例 \theexample:\ #1} \hspace{0.5em}}{\hspace{\stretch{4}} \par  }
\newcommand{\da}[1]{\noindent {\kaishu \textbf{解:}#1}}

\biaoti{ 不等式技巧和题型}
\fubiaoti{}
 \maketitle
\section{均值不等式}
\subsection{基本不等式的常用结论}
\begin{enumerate}
\item 若$ a,~b\inR $,则$ a^2+b^2\ge 2ab $\ $\left(\text{当且仅当}~a=b~\text{时取}“=”\right)$;
\item 若$ a,~b\inR $且$ a>0,~b>0 $,则$ a+b\ge 2\sqrt{ab} $\ $\left(\text{当且仅当}~a=b~\text{时取}“=”\right)$;
\item 若$ a,~b\inR $且$ a>0,~b>0 $,则$ab\le\left(\dfrac{a+b}{2}\right)^2$\ $\left(\text{当且仅当}~a=b~\text{时取}“=”\right)$;
\item 若$ a,~b\inR $,则$ \left(\dfrac{a+b}{2}\right)^2\le \dfrac{a^2+b^2}{2} $\ $\left(\text{当且仅当}~a=b~\text{时取}“=”\right)$;
\item 若$ ab>0 $,则$ \dfrac{a}{b}+\dfrac{b}{a}\ge 2 $\ $\left(\text{当且仅当}~a=b~\text{时取}“=”\right)$;
\item 若$ ab\ne 0 $,则$ \abs{\dfrac{a}{b}+\dfrac{b}{a}}\ge 2 $ 即$ \dfrac{a}{b}+\dfrac{b}{a}\ge 2 $或$ \dfrac{a}{b}+\dfrac{b}{a}\le 2 $\ $\left(\text{当且仅当}~a=b~\text{时取}“=”\right)$;
\item 若$ x>0 $,则$ x+\dfrac{1}{x}\ge2 $\ $\left(\text{当且仅当}~x=1~\text{时取}“=”\right)$;
\item 若$ x<0 $,则$ x+\dfrac{1}{x}\le -2 $\ $\left(\text{当且仅当}~x=-1~\text{时取}“=”\right)$;
\item 若$ x\ne 0 $,则$ \abs{x+\dfrac{1}{x}}\ge 2 $,即$ x+\dfrac{1}{x}\ge2 $或$ x+\dfrac{1}{x}\le -2 $\ $\left(\text{当且仅当}~x=\pm 1~\text{时取}“=”\right)$;
\end{enumerate}
\subsection{注意事项}
{\kaishu  \begin{enumerate}[1)] 
\item 当两个正数的积为定值时,可以求它们的和的最小值,当两个正数的和为定值时,可以求它们的积的最大值,所谓“积定和最小,和定积最大”;
\begin{enumerate}[(1)]
\item 如果积$ xy $是定值$ p $,那么当且仅当$x=y  $时,$ x+y $有最小值,是$ 2\sqrt{p} .$(简记:积定和最小)
\item 如果和$ x+y $是定值$ s $,那么当且仅当$ x=y $时,$ xy $有最大值,是$ \dfrac{s^2}{4}. $
\end{enumerate}
\item 求最值的条件:“一正、二定、三取等”.
\begin{itemize}
\item 正:两数均为正数;
\item 定:两数的和(或积)为定值;
\item 等:两数相等的条件存在.
\end{itemize}
\item 若无明显“定值”,则用配凑的方法,使和为定值或积为定值,当多次使用基本不等式时,一定要注意每次是否能保证等号成立,并注意取等号的条件的一致性.
\item 重要不等式总结:\begin{enumerate}[(1)]
\item $ \sqrt{\dfrac{a^2+b^2}{2}}\ge \dfrac{a+b}{2}\ge \sqrt{ab}\ge \dfrac{2}{\dfrac{1}{a}+\dfrac{1}{b}}~(a,b>0) $
\item $\dfrac{a^2+b^2}{2}\ge \left(\dfrac{a+b}{2}\right)^2\ge ab ~(a,b\inR)$.
\end{enumerate}

\end{enumerate}
}
\subsection{解题技巧}
\begin{enumerate}[\textbf{技巧} 1)]
\item \textbf{凑项}
\begin{example}
已知$ x<\dfrac{5}{4} $,求函数$ y=4x-2+\dfrac{1}{4x-5} $的最大值.
\end{example}
\da{因为$ 4x-5<0 $,所以首先要“调整”符号,又$ (4x-2)\left(\dfrac{1}{4x-5}\right)$不是常数,所以要对$ 4x-2 $拆项、凑项,有:\[ y=4x-2+\dfrac{1}{4x-5} =-\left(5-4x+\dfrac{1}{5-4x}\right)+3\le -2+3=1\]
当且仅当$ 5-4x=\dfrac{1}{5-4x} $,即$ x=1 $时,等号成立.
}
\item \textbf{凑系数}
\begin{example}
当$0<x<4$时,求$ y=x(8-2x) $的最大值.
\end{example}
\da{由$0<x<4$知,$ 8-2x>0 $,利用基本不等式求最值,必须为和为定值或者积为定值.此题为两个式子积的形式,但其和不是定值.~注意到$ 2x+(8-2x) $为定值,故只需将$ y=x(8-2x) $凑上一个系数即可.
\[ y=x(8-2x)=\dfrac{1}{2}\left[2x\bm\cdot\left(8-2x\right)\right]\le \dfrac{1}{2}\left(\dfrac{2x+8-2x}{2}\right)^2=8 \]
当且仅当$2x=8-2x  $,即$ x=2 $时,$ y=x(8-2x) $取得最大值为8
}
\item \textbf{分离}
\begin{example}
求$y=\dfrac{x^2+7x+10}{x+1}~(x>-1)$的值域.
\end{example} 
\da{本题看似无法运用基本不等式,不妨将分子配方凑出含有$ (x+1) $的项,再将其分离\[ y=\dfrac{x^2+7x+10}{x+1}=\dfrac{(x+1)^2+5(x+1)+4}{x+1}=(x+1)+\dfrac{4}{x+1}+5 \]当$ x>-1 $即$ x+1>0 $时,$$ y\ge 2\sqrt{(x+1)\dfrac{4}{x+1}}+5=9 $$当且仅当$x=1  $时取$ “=” $.
}
\newpage
\item \textbf{换元}
\begin{example}
求$y=\dfrac{x^2+7x+10}{x+1}~(x>-1)$的值域.
\end{example}
\da{本题还可以用换元法解答,令$ t=x+1 $,化简原式然后分离求最值.
\[ y=\dfrac{(t-1)^2+7(t-1)+10}{t}=\dfrac{t^2+5t+4}{t}=t+\dfrac{4}{t}+5 \]当$ x>-1 $即$ t=x+1>0 $时,\[ y\ge 2\sqrt{t\times\dfrac{4}{t}}+5=9 \](当$ t=2 $即$ x=1 $时取$ “=”. $)
}\par 
\textbf{注:}分式函数求最值,通常直接将分子配凑后将式子分开或者将分母换元后将式子分开再利用不等式求最值.即化为$ y=mg(x)+\dfrac{A}{g(x)}+B~(A>0,~B>0) $,$ g(x) $恒正或恒负的形式,然后运用基本不等式.
\item \textbf{等号取不到时,结合$ f(x)=x+\dfrac{a}{x} $的单调性}
\begin{example}
求函数$ y=\dfrac{x^2+5}{\sqrt{x^2+4}} $的值域.
\end{example}
\da{令$ \sqrt{x^2+4}=t~(t\ge 2) $,则$$ y=\dfrac{x^2+5}{\sqrt{x^2+4}}=\sqrt{x^2+4} +\dfrac{1}{\sqrt{x^2+4}}=t+\dfrac{1}{t}~(t\ge 2)$$
因为$ t>0,~t\cdot \dfrac{1}{t}=1 $但是$ t=\dfrac{1}{t} $的解$ t=\pm 1 $不在区间$ \left[2,+\infty\right) $内,故等号不成立,考虑单调性:
因为$ y=t+\dfrac{1}{t} $在区间$ \left[1,+\infty\right) $单调递增,所以在其子区间$ \left[2,+\infty\right)  $为单调递增函数,故$ y\ge \dfrac{5}{2}. $
所以所求函数的值域为$ \left[\dfrac{5}{2},+\infty\right) .$}
\item \textbf{整体代换:}
多次连用最值定理求最值时,注意取等号的条件的一致性.
\begin{example}
已知$ x>0,~y>0,~ $且$ \dfrac{1}{x}+\dfrac{9}{y}=1 $,求$ x+y $的最大值.
\end{example}
\da{因为$x>0,~y>0,~\dfrac{1}{x}+\dfrac{9}{y}=1$\\所以$$x+y=(x+y)\left(\dfrac{1}{x}+\dfrac{9}{y}\right)=\dfrac{y}{x}+\dfrac{9x}{y}+10\ge 6+10=16$$当且仅当$ \dfrac{y}{x}=\dfrac{9x}{y} $时,等号成立,又$ \dfrac{1}{x}+\dfrac{9}{y}=1 $\\可得$ x=4,~y=12 $时,$ (x+y)_{min} =16.$
}
\item \textbf{取平方}
\begin{example}
已知$ x,~y $为正数,$ 3x+2y=10 $,求函数$ W=\sqrt{3x}+\sqrt{2y} $的最值.
\end{example}
\da{解法一:若利用算术平均与平方平均之间的不等关系,$ \dfrac{a+b}{2}\le \sqrt{\dfrac{a^2+b^2}{2}},~ $本题就很简单:\[\sqrt{3x}+\sqrt{2y}\le \sqrt{2}\sqrt{\left(\sqrt{3x}\right)^2+\left(\sqrt{2y}\right)^2}=\sqrt{2} \sqrt{3x+2y}=2\sqrt{5} \]
解法二:条件与结论均为和的形式,设法直接用基本不等式,应通过平方化函数式为积的形式,再向“和为定值”条件靠拢.
\[ W>0,~W^2=3x+2y+2\sqrt{3x}\bm{\cdot}\sqrt{2y}\le 10+\left(\sqrt{3x}\right)^2 \bm{\cdot}\left(\sqrt{2y}\right)^2=10+(3x+2y)=20\]
$ W\le \sqrt{20}=2\sqrt{5} $
}
\end{enumerate}
\subsection{不等式举例}
\begin{enumerate}[1)]
\item 已知$ x>0,~y>0 ,~x+2y+2xy=8,~$则$ x+2y $的最小值是\xx
\onech{$ 3$}{$ 4$}{$ \dfrac{9}{2}$}{$ \dfrac{11}{2}$}
\item 若$ 2^x+2^y=1 ,~$则$ x+y $的取值范围是\xx
\onech{$ \left[0,~2\right]$}{$ \left[-2,~0\right]$}{$ \left[-2,~+\infty\right)$}{$ \left(-\infty,-2\right]$}
\item 设正实数$ x,~y,~z $满足$ x^2-3xy+4y^2-z=0 $.则当$ \dfrac{xy}{z} $取得最大值时,$ \dfrac{2}{x}+\dfrac{1}{y}-\dfrac{2}{z} $的最大值为\xx
\onech{$0 $}{$ 1$}{$ \dfrac{9}{4}$}{$ 3$}
\item 设正实数$ x,~y,~z $满足$ x^2-3xy+4y^2-z=0 $.则当$ \dfrac{z}{xy} $取得最小值时,$ x+2y-z $的最大值为\xx
\onech{$ 0$}{$ \dfrac{9}{8}$}{$ 2$}{$ \dfrac{9}{4}$}
\item 若正数$ x,~y $满足$ x+3y=5xy,~ $则$ 3x+4y $的最小值为\xx
\onech{$ \dfrac{24}{5}$}{$ \dfrac{28}{5}$}{$ 5$}{$ 6$}
\item 设$ a,b\inR,a^2+2b^2=6, $则$ a+b $的最小值是\xx
\onech{$ -2\sqrt{2}$}{$ -\dfrac{5\sqrt{3}}{3}$}{$-3 $}{$ -\dfrac{7}{2}$}
\item 若$ 2a+b=3 $,求$ 4^a+2^b $的最小值为\tk.
\item 已知$ \log_2a+\log_2b\ge1 ,~$则$ 3^a+9^b $的最小值为\tk.
\item 设$ x,~y $为正数,则$ (x+y)\left(\dfrac{1}{x}+\dfrac{4}{y}\right) $的最小值为\tk.
\item 若实数$ x,~y $满足$ x^2+y^2+xy=1,~ $则$ x+y $的最大值是\tk.
\item 设$ a>0,~b>0,~ $若$ \sqrt{3} $是$ 3^a,~3^b $的等比中项,求$ \dfrac{1}{a}+\dfrac{1}{b} $的最小值为\tk.
\item 已知实数$ a,~b,~c $满足$ a+b+c=0,~a^2+b^2+c^2=1. $则$ a $的最大值为\tk.
\item 已知$ \dfrac{2}{x}+\dfrac{3}{y}=2~(x>0,y>0 ) $,则$xy$的最小值是\tk.
\item 设$ a,b,c\in \mathbf{R^+} $,求证:$ \dfrac{bc}{a}+\dfrac{ca}{b}+\dfrac{ab}{c}\ge a+b+c. $
\item 为了促销某电子产品,商场进行降价,设$ m>0,~n>0,~m\ne n,~ $有三种降价方案:\\
方案\ding{192}: 先降$ m\%,~ $再降$n\%  $;\\
方案\ding{193}:先降$ \dfrac{m+n}{2}\% $,再降$ \dfrac{m+n}{2}\% $\\
方案\ding{194}:一次性降价$ \left(m+n\right) \%$.\\
则降价幅度最小的方案是\tk.(填出正确的序号)
\end{enumerate}




\newpage
\section{线性规划}
\subsection{线性规划问题一般程序}
\begin{enumerate}[1)]
\item 确定由二元一次不等式表示的平面区域;
\item 令$ z=0 $画直线$l_0:~ax+by=0  $;
\item 平移直线$ l_0 $寻找使直线$ y=-\dfrac{a}{b}x+\dfrac{z}{b} $截距取最值(最大或最小)的位置(最优解);
\item 将最优解坐标代入线性目标函数$ z=ax+by $求出最值.
\end{enumerate}
\subsection{注意问题}
\begin{enumerate}[1)]
\item 解决线性规划问题要特别关注线性目标函数$ z=ax+by $中$ b $的符号,若$ b>0 $,则使$ y=-\dfrac{b}{a}x+\dfrac{z}{b} $的截距取最大(小)值的点,可使目标函数$ z=ax+by $取最大(小)值;若$b<0$,则使$ y=-\dfrac{b}{a}x+\dfrac{z}{b} $的截距取最大(小)值的点,可使目标函数$ z=ax+by $取最小(大)值,$ b<0 $的情况是很容易被忽略的情况.
\item 解决线性规划问题首先考虑可行域,若为封闭区域,则一般在区域顶点处取得最大或最小值.
\item 避免思维误区:\begin{itemize}
\item 忽视平面区域是否包括边界,一般最优解都处于平面区域的边界顶点处,若平面区域不包括边界, 则可能不存在最值;
\item 忽视对$ z=ax+by $中$ b $的符号的区分
\end{itemize}
\end{enumerate}
\subsection{常考变式}

\begin{enumerate}[1)]
\item 形如$ \dfrac{y-b}{x-a} $的式子,表示动点$ M(x,~y) $和定点$ N(a,~b) $连线的斜率$ k $;
\item 形如$ \sqrt{(x-a)^2+(y-b)^2} $的式子,表示动点$ M(x,~y) $到定点$ N(a,b) $的距离;
\item 形如$ \dfrac{\left|ax+by+c\right|}{\sqrt{a^2+b^2}} $的式子,表示动点$ M(x,~y) $到直线$ ax+by+c=0 $的距离$ d $;而$ \left|ax+by+c\right| $表示$ \sqrt{a^2+b^2}d .$
\end{enumerate}
\subsection{线性规划举例}
\subsubsection{最值问题}
\begin{example}
若变量$ x,~y $满足约束条件$ \begin{dcases}
x+y\le 2,\\
x\ge 1,\\
y\ge 0.
\end{dcases} $则$ z=2x+y $的最大值是\tk;最小值是\tk.
\end{example}
\subsubsection{距离问题}
\begin{example}
设$ D $为不等式组$ \begin{dcases}
x\ge 0,\\
2x-y\le 0,\\
x+y-3\le 0
\end{dcases} $表示的平面区域,区域$ D $上的点与点$ (1,0) $之间的距离的最小值为\tk.
\end{example}
\subsubsection{斜率问题}
\begin{example}
在平面直角坐标系$xOy$中,$ M $为不等式组$ \begin{dcases}
2x-y-2\ge 0,\\
x+2y-1\ge 0,\\
3x+y-8\le 0.
\end{dcases} $所表示的区域上一动点,则直线$ OM $斜率的最小值为\tk.
\end{example}

\subsubsection{参数问题}
\begin{example}
设$ z=kx+y $,其中实数$ x,y $满足$\begin{dcases}
x+y-2\ge 0,\\
x-2y+4\ge 0,\\
2x-y-4\le 0.
\end{dcases}$若$ z $的最大值为$ 12,~ $则实数$ k= $\tk.
\end{example}
\subsubsection{范围问题}
\begin{example}
设关于$ x,y $的不等式组$\begin{dcases}
2x-y+1>0,\\
x+m<0,\\
y-m>0.
\end{dcases}$表示的平面区域内存在点$ P(x_0,~y_0) $满足$ x_0-2y_0=2,~ $求得$ m $的取值范围是\tk.
\end{example}

\subsubsection{创新问题}
\begin{example}
给定区域$ D:\begin{dcases}
x+4y\ge 4,\\
x+y\le 4,\\
x\ge 0.
\end{dcases} $令点集$ T=\left\{(x_0,y_0\in D|x_0,y_0\in \mathbf{Z},~(x_0,~y_0)\text{是}z=x+y\right.$上取得 最大值或最小值的点$\big\} $,则$ T $中的点共确定\tk 条不同的直线.
\end{example}
\subsection{练习}
\begin{questions}
\qs 若$ x,~y $满足$\begin{dcases}
x+y-2\ge 0,\\
kx-y+2\ge 0,\\
y\ge 0.
\end{dcases}$且$ z=y-x $的最小值为$ -4,~ $则$ k $的值为\xx
\onech{$ 2$}{$-2 $}{$ \dfrac{1}{2}$}{$ -\dfrac{1}{2}$}

\qs 若直线$ y=2x $上存在点$ (x,y) $满足约束条件$\begin{dcases}
x+y-3\le0,\\
x-2y-3\le0,\\
x\ge m.
\end{dcases}$则实数$ m $的最大值是\xx
\onech{$ -1$}{$ 1$}{$ \dfrac{3}{2}$}{$ 2$}
\qs 设$x,~y$满足约束条件$\begin{dcases}
y-x\le1,\\
x+y\le3,\\
y\ge m.
\end{dcases}$若$ z=x+3y $的最大值与最小值之差为$ 7 $,则实数$ m $的取值范围是\xx
\onech{$ \dfrac{3}{2}$}{$ -\dfrac{3}{2}$}{$ \dfrac{1}{4}$}{$ -\dfrac{1}{4}$}
\qs 若$ x,y $满足$\begin{dcases}
x+y\ge0,\\
x\ge1,\\
x-y\ge0.
\end{dcases}$则下列不等式恒成立的是\xx
\onech{$ y\ge1$}{$ x\ge2$}{$ x+2y+2\ge0$}{$ 2x-y+1\ge0$}
\qs 设关于$ x,y $的不等式组$ \begin{dcases}
2x-y+1>0\\x+m<0\\y-m>0
\end{dcases} $
表示的平面区域内存在点$ P(x_0,y_0) $满足$ x_0-2y_0=2 $,求得$ m $的取值范围是\xx
\onech{$\left(-\infty,-\dfrac{4}{3}\right)$}{$\left(-\infty,-\dfrac{1}{3}\right)$}{$\left(-\infty,-\dfrac{2}{3}\right)$}{$\left(-\infty,-\dfrac{5}{3}\right)$}
\qs 已知$ a>0,~x,~y $满足约束条件$ \begin{dcases}
x\ge 1,\\
x+y\le 3,\\
y\ge a(x-3).
\end{dcases} $若$ z =2x+y$的最小值为$ 1,~ $则$ a= $\xx
\onech{$ \dfrac{1}{4}$}{$\dfrac{1}{2} $}{$ 1$}{$ 2$}
\qs 设$x,~y$满足约束条件$\begin{dcases}
x-y\ge0,\\
x+y\le2,\\
y\ge0.
\end{dcases}$若$ z=ax+y $的最大值为$ 4 $,则$ a= $\xx
\onech{$ 3$}{$ 2$}{$ -2$}{$ -3$}
\qs 设$ x,~y $满足约束条件$\begin{dcases}
x+y-2\le0,\\
x-2y-2\le0,\\
2x-y+2\ge0.
\end{dcases}$若$ z=y-ax $取得最大值的最优解\CJKunderdot{不唯一},则实数$ a $的值为\xx
\onech{$ \dfrac{1}{2}\text{或}-1$}{$ 2\text{或}~\dfrac{1}{2}$}{$ 2\text{或}~1$}{$ 2\text{或}-1$}
\qs 若函数$ y=\log_2x $的图象上存在点$ (x,y) $满足约束条件$ \begin{dcases}
x+y-3\le0,\\
2x-y+2\ge0,\\
y\ge m.
\end{dcases} $ 则实数$ m $的最大值为\xx
\onech{$ \dfrac{1}{2}$}{$ 1$}{$ \dfrac{3}{2}$}{$ 2$}
\qs 设不等式组$ \begin{dcases}
x+y-11\ge 0\\3x-y+3\ge 0\\5x-3y+9\ge 0
\end{dcases} $表示的平面区域为$ D $,若指数函数$ y=a^x $的图象上存在区域$ D $上的点,则$ a $的取值范围是\xx
\onech{$\left(1,3\right]$}{$\left[2,3\right]$}{$\left(1,2\right]$}{$\left[3,+\infty\right)$}

\qs 不等式组$ \Bigg\{ \begin{aligned}
x+y\ge1\\
x-2y\le 4
\end{aligned} $的解集记为$ D $.下列四个命题中真命题是\xx\\
$ \begin{aligned}
&p_1:\forall(x,y)\in D,x+2y\ge-2, &p_2&:\forall(x,y)\in D,x+2y\ge2,\\
&p_3:\forall(x,y)\in D,x+2y\le 3, &p_4&:\forall(x,y)\in D,x+2y\le-1 
\end{aligned}$\\
\onech{$p_2,p_3$}{$p_1,p_4$}{$p_1,p_2$}{$p_1,p_3$}
\qs 设$x,~y$满足约束条件$\begin{dcases}
x+y\ge a,\\
x-y\le -1.
\end{dcases}$且$ z=x+ay $的最小值为$ 7 $,则$ a= $\xx
\onech{$ -5$}{$ 3$}{$ -5\text{或}3$}{$ 5\text{或}-3$}
\qs 已知$ a,b $是正数,且满足$ 2<a+2b<4, ~$那么$ \dfrac{b+1}{a+1} $的取值范围是\xx
\twoch{$ \left(\dfrac{1}{5},3\right)$}{$ \left(\dfrac{1}{3},2\right)$}{$\left(\dfrac{1}{5},2\right) $}{$\left(\dfrac{1}{3},3\right) $}
\qs 已知$ 6 $枝玫瑰与$ 3 $枝康乃馨的价格之和大于$ 24 $元,而$ 4 $枝玫瑰与$ 4 $枝康乃馨的价格之和小于$ 20 $元,那么$ 2 $枝玫瑰和$ 3 $枝康乃馨的价格的比较结果是\xx
\twoch{$ 2$枝玫瑰的价格高}{$ 3$枝康乃馨的价格高}{价格相同}{不确定}







\qs 若变量$x,y$满足约束条件$\begin{dcases}
y\le x,\\
x+y\le 4,\\
y\ge k.
\end{dcases}$且$z=2x+y$的最小值为$ -6 $,则$ k= $\tk.
\qs 若$  x,~y $满足约束条件$\begin{dcases}
x-1\ge 0,\\
x-y\le 0,\\
x+y-4\le 0.
\end{dcases}$则$ \dfrac{y}{x} $的最大值为\tk.
\qs 实数$ a,~b $满足$ 0<a\le 2,~b\ge 1 $,若$ b\le a^2 $,则$ \dfrac{b}{a} $的取值范围是\tk.
\qs 已知实数$ u,v,x,y $满足$ u^2+v^2=1,~\begin{dcases}
x+y-1\ge 0,\\
x-2y+2\ge 0,\\
x\le 2.
\end{dcases} $则$ z=ux+vy $的最大值是\tk.

\qs 满足约束条件$ \left|x\right|+2\left|y\right|\le 2 $的目标函数$ z=y-x $的最小值是\tk.
\qs 已知$ x\ge0,y\ge0 ,$且$ x+y=1 ,$则$ x^2+y^2 $的取值范围是\tk.
\qs 已知变量$ x,~y $满足$\begin{dcases}
x-4y+3\le 0,\\
x+y-4\le 0,\\
x\ge 1.
\end{dcases}$点$ (x,~y) $对应的区域的面积为\tk;$ \dfrac{x^2+y^2}{xy} $的取值范围为\tk.
\qs 设$ m>1 $,在约束条件$\begin{dcases}
y\ge x,\\
y\le mx,\\
x+y\le1. 
\end{dcases}$下,目标函数$ z=x+5y $的最大值为$ 4 $,则$ m $的值是\tk.

\end{questions}
\end{document}