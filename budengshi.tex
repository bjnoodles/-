
\section{导数证明不等式问题}

\newcounter{example}
\renewcommand{\theexample}{\arabic{example}}
\newenvironment{example}[1][]{ \refstepcounter{example} \textbf{例\theexample:#1} \hspace{0.5em}}{\hspace{\stretch{1}} }

\subsection{问题分析}
\subsubsection{恒成立问题}
\begin{enumerate}[1)]
\item $ \forall x\in D, ~$均有$ f(x)>A $恒成立,则$ f(x)_{min}>A $;
\item $ \forall x\in D, ~$均有$ f(x)<A $恒成立,则$ f(x)_{max}<A $;
\item  $ \forall x\in D, ~$ 均有$ f(x)>g(x) $恒成立,则$ F(x)=f(x)-g(x) >0$恒成立,即$ F(x)_{min}>0 $;
\item  $ \forall x\in D, ~$ 均有$ f(x)<g(x) $恒成立,则$ F(x)=f(x)-g(x) <0$恒成立,即$ F(x)_{max}<0 $;
\item $\forall x_1\in D,~$$\forall x_2\in E,~$均有$ f(x_1)>g(x_2) $恒成立,则$ f(x)_{min}>g(x)_{max} $;
\item $\forall x_1\in D,~$$\forall x_2\in E,~$均有$ f(x_1)<g(x_2) $恒成立,则$ f(x)_{max}<g(x)_{min} $;
\end{enumerate}
\subsubsection{存在性问题}
\begin{enumerate}[1)]
\item $ \exists x_0\in D ,~$使得$ f(x_0) >A$成立,则$ f(x)_{max}>A; $ 
\item $ \exists x_0\in D ,~$使得$ f(x_0) <A$成立,则$ f(x)_{min}>A; $ 
\item $\exists x_0 \in D,~$使得$ f(x_0)>g(x_0) $成立,设$ F(x)=f(x)-g(x) ,~$则$ F(x)_{max}>0 $;
\item $\exists x_0 \in D,~$使得$ f(x_0)<g(x_0) $成立,设$ F(x)=f(x)-g(x) ,~$则$ F(x)_{min}<0 $;
\item $\exists x_1\in D,~$$\exists x_2\in E,~$使得$ f(x_1)>g(x_2) $成立,则$ f(x)_{max}>g(x)_{min} $;
\item $\exists x_1\in D,~$$\exists x_2\in E,~$使得$ f(x_1)<g(x_2) $成立,则$ f(x)_{min}<g(x)_{max} $;
\end{enumerate}
\subsubsection{相等问题}
{\kaishu \large \noindent 若$f(x),~g(x)$的值域分别是$ A,~B $则:}
\begin{enumerate}[1)]
\item $\forall x_1\in D,~$$\exists x_2\in E,~$使得$ f(x_1)=g(x_2) $成立,则$ A\subseteq B $;
\item $\exists x_1\in D,~$$\exists x_2\in E,~$使得$ f(x_1)=g(x_2) $成立,则$ A\cap B\ne \varnothing $;
\end{enumerate}
\subsubsection{恒成立与存在性综合问题}
\begin{enumerate}[1)]
\item $\forall x_1\in D,~$$\exists x_2 \in E,~$使得$f(x_1)>g(x_2) $成立,则$ f(x)_{min}>g(x)_{min} $
\item $\forall x_1\in D,~$$\exists x_2 \in E,~$使得$ f(x_1)<g(x_2) $成立,则$ f(x)_{max}<g(x)_{max} $
\end{enumerate}
\subsection{解法举例}
\subsubsection{分离参数法}
分离参数法确定不等式$ f(x,\lambda)\ge 0 ~(x\in D,\lambda\text{为实数})$恒成立问题中的参数$ \lambda $的取值范围的基本步骤:
\begin{enumerate}[1)]
\item 将参数和变量分离,即化为$ g(\lambda)\ge f(x) \text{或} g(\lambda)\le f(x)$恒成立的形式;
\item 求$ f(x) $在$ x\in D $上的最大(小)值;
\item 解不等式$ g(\lambda)\ge f(x)_{max} \text{或}g(\lambda)\le f(x)_{min}$,得到$ \lambda $的取值范围.
\end{enumerate}
\begin{example}
已知函数$f(x)=ax+x\ln x$的图象在$ x=e $处的切线斜率为$ 3 .$
\end{example}
\begin{enumerate}[(1)]
\item 求实数$a$的值;
\item 若$f(x)\le kx^2$对任意$ x>0 $恒成立,求实数$k$的取值范围.
\end{enumerate}
\begin{proof}[分析]
\begin{enumerate}[(1)]
\item \label{it1}由$f'(x)=a+\ln x+1$结合条件$ f(x)=ax+x\ln x $的图象在点$ x=e $处的切线的斜率为$ 3 ,~$可知$ f'(e)=3,~ $建立方程解得$ a=1 $
\item 要使$ f(x)\le kx^2 $对任意的$ x\ge 0 $恒成立,只需$ k\ge \left(\dfrac{f(x)}{x^2}\right)_{max} $即可,\\而由(\ref{it1})可知$ f(x)=x+x\ln x $,
所以问题等价于求函数$ g(x)=\dfrac{1+\ln x}{x} $的最大值.\\求导可得:
\[ g'(x)=\dfrac{\dfrac{1}{x}\cdot x-(1+\ln x)}{x^2}=-\dfrac{\ln x}{x^2} \]
令$ g'(x)=0 $解得$ x=1 $.经过检验可得$ g(x) $在$ x=1 $处取得最大值$ g(1)=1 $,故$ k\ge 1 $即为所求.
\end{enumerate}
\end{proof}
\subsubsection{主参换位法}
\begin{example}
已知函数$f(x)=\ln (e^x+a)~(a\text{为常数})$是实数集$ \mathbf{R}$上的奇函数,函数$ g(x)=\lambda f(x)+\sin x $是区间$ \left[-1,1\right] $ 上的减函数.\end{example}
\begin{enumerate}[(1)]
\item 求$ a $的值;
\item 若$ g(x)\le t^2+\lambda t+1 $在$ \left[-1,1\right] $上恒成立,求$ t $的取值范围.
\end{enumerate}

\begin{proof}[解析]
\begin{enumerate}[(1)]
\item \label{it2}a=1;
\item 由(\ref{it2})知,$ f(x)=x,~g(x)=\lambda x+\sin x $.\\因为$ g(x) $在$ \left[-1,~1\right] $上单调递减,所以有\[ g'(x)=\lambda +\cos x\le 0. \]
所以$ \lambda \le -\cos x $在$ \left[-1,~1\right] $上恒成立\\
所以$$ \lambda \le -1,~g(x)_{max} =-\lambda-\sin 1$$只需要$ -\lambda -\sin 1\le t^2+\lambda t+1 $即$ (t+1)\lambda +t^2+\sin 1 +1\ge 0~(\lambda\le -1) $恒成立.\\
可令$ f(\lambda)=(t+1)\lambda +t^2+\sin 1+1\ge 0~(\lambda \le -1) ,~$则有:\[ \Bigg\{\begin{aligned}
&t+1\le 0,\\
&-t-1+t^2+\sin 1+1\ge 0.
\end{aligned}\]解得$ t\le -1. $
\end{enumerate}
\end{proof}
\subsubsection{存在性之常用模型及方法}
\begin{example}
已知函数$f(x)=a\ln x+\dfrac{1-a}{2}x^2-bx,~a\inR$且$ a\ne 1 .~$曲线$ y=f(x) $在点$\left(1,f(1)\right)  $处的切线的斜率为$0$.
\end{example}
\begin{enumerate}[(1)]
\item 求$b$的值;
\item 若存在$ x\in \left[1,+\infty\right),~ $使得$ f(x)<\dfrac{a}{a-1} ,~$求$ a $的取值范围.
\end{enumerate}
\begin{proof}[解析]
\begin{enumerate}[(1)]
\item $ b =1$;
\item 若存在$ x\in \left[1,+\infty\right) $使得不等式$ f(x)<\dfrac{a}{a-1} $成立,只需$ \dfrac{a}{a-1}>f(x)_{min} $即可.因此可通过探求$ f(x) $的单调性进而求得$f(x)$的最小值,进而得到关于$ a $的不等式即可.\\
而由(1)可知$ f(x)=a\ln x+\dfrac{1-a}{2}x^2-x ,~$则\[f'(x)=\dfrac{(x-1)\left[(1-a)x-a\right]}{x}\]对$ a $的取值范围进行分类讨论并判断$f(x)$的单调性,\\从而可以解得$ a $的取值范围是$ \left(-\sqrt{2}-1,\sqrt{2}-1\right)\cup(1,+\infty). $
\end{enumerate}
\end{proof}
\newpage
 \subsection{习题}
\begin{questions}

\question
(2013新课标理)已知函数$f(x)=e^x-\ln (x+m)$.
\begin{parts}                    
\part 设$x=0$是$f(x)$的极值点,求$m$,并讨论$f(x)$的单调性;
\part 当$m\le 2$时,证明$f(x)>0$
\end{parts}
\kb 
\qs
(2012新课标理)已知函数$f(x)$满足$f(x)=f'(1)e^{x-1}-f(0)x+\dfrac{1}{2}x^2$.
\begin{parts}
\part 求$f(x)$的解析式及单调区间;
\part 若$f(x)\ge \dfrac{1}{2}x^2+ax+b$,求$(a+1)b$的最大值.
\end{parts}
\kb 
\question
(2015理)已知函数$f(x)=\ln\dfrac{1+x}{1-x}$.
\begin{parts}
\part[3]求曲线$y=f(x)$在点$\left(0,f(0)\right)$处的切线方程;
\part[5]求证:当$x\in(0,1)$时,$f(x)>2\left( x+\dfrac{x^3}{3}\right) $;
\part[5]设实数$k$使得$f(x)>k\left( x+\dfrac{x^3}{3}\right)$对$x\in(0,1)$恒成立,求$k$的最大值.
\end{parts}
\kb
\question
(2014理)已知函数$f(x)=x\cos x-\sin x,x\in\left[0,\dfrac{\pi}{2}\right]$.
\begin{parts}
\part[5]求证:$f(x)\le0$;
\part[8]若$a$<$\dfrac{\sin x}{x}<b$在$(0,\dfrac{\pi}{2})$上恒成立,求$a$的最大值和$b$的最小值
\end{parts}
\kb 
\question
(2014新课标理)设函数$f(x)=ae^x\ln x+\dfrac{be^{x-1}}{x}$,曲线$y=f(x)$在点$(1,f(1))$处的切线方程为$y=e(x-1)+2$.
\begin{parts}
\part 求$a,b$;
\part 证明:$f(x)>1$.
\end{parts}
\kb
\qs 已知函数$f(x)=(x+1)\ln x-a(x-1)$.
\begin{parts}
\part 当$a=4$时,求曲线$y=f(x)$在$(1,f(1))$处的切线方程;
\part 若当$x\in (1,+\infty)$时,$f(x)>0$,求$a$的取值范围.
\end{parts}
\kb
\qs
设函数$f(x)=e^x-ax-2$.
\begin{parts}
\part 求$ f(x)  $的单调区间;
\part 若$ a=1 $,$ k $为整数,且当$ x>0 $时,$ (x-k)f'(x)+x+1>0 $,求$ k $的最大值.
\end{parts}
\kb 
\qs 已知函数$f(x)=\ln(kx)+\dfrac{1}{x}-k~(k>0).$
\begin{parts}
\part 求$f(x)$的单调区间;
\part 对任意$ x\in \left[\dfrac{1}{k},\dfrac{2}{k}\right],~ $都有$ x\ln (kx)-kx+1\le mx, $求$ m $的取值范围.
\end{parts}
\kb 
\qs 已知函数$f(x)=\dfrac{x+1}{e^x},~A(x_1,m),~B(x_2,m)$是曲线$ y=f(x) $上的两个不同的点.
\begin{parts}
\part 求$f(x)$的单调区间,并写出实数$ m $的取值范围
\part 证明:$ x_1+x_2>0 $
\end{parts}
\kb 
\qs 已知函数$f(x)=\dfrac{1}{2}ax^2-(2a+1)x+2\ln x~(a\inR),~g(x)=x^2-2x$,若对任意的$ x_1\in\left(0,2\right] $,均存在$ x_2\in\left(0,2\right] $使得$ f(x_1)<g(x_2) $,求$ a $的取值范围.





\end{questions}
